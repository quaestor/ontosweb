% vim: set filetype=tex:
\newcommand{\defin}[2]{#1 $\sqsubseteq$ #2}
\newcommand{\ando}{$\sqcap$\hspace{1pt}}
\newcommand{\kripke}[1]{\ensuremath{\mathcal{#1}}}
\newcommand{\relation}[1]{\ensuremath{\mathcal{#1}}}
\newcommand{\dia}{\Diamond}

\aufgabe{1}{Anwendung des Tableaukalküls}

\angabe{%
Verwenden Sie den Tableaukalkül, um folgende Formeln auf Erfüllbarkeit zu prüfen:
\begin{equation*}
\Box(\dia p) \wedge \Box(p \rightarrow{} (\dia q \vee \dia r)) \wedge
\dia p \wedge \dia \neg p \wedge \Box \Box ((q \vee r) \rightarrow{} \neg p)
\wedge \Box (\neg p \rightarrow{} \Box \neg p)
\label{erfbar1}
\end{equation*}
\begin{equation*}
	\Box(p \rightarrow{} (\dia q \vee \dia r)) \wedge (\dia p \vee \dia \neg p)
	\wedge \Box(\neg p \rightarrow{} \Box \neg p) \wedge \Box \Box (p \wedge
	\neg r)
	\label{erfbar2}
\end{equation*}
}
%%1. Formel
Die erste Formel sei verkürzt wie folgt dargestellt (nach Anwedung von Umformungsregeln
für Implikation, Diamond und De Morgan'sche Regel, sowie Anwendung der
$(\wedge)$-Regel auf oberster Ebene sowie $(\neg \neg)$-Regel des Kalküls); anschließend
wird der Kalkül wie farbig markiert durchgeführt.

\[\Box(\dia p)
	\wedge \Box \underbrace{(p \rightarrow{} (\dia q \vee \dia r))}_{\coloneqq \phi}
\wedge \dia p
\wedge \dia \neg p
\wedge \Box \underbrace{\Box ((q \vee r) \rightarrow{} \neg p)}_{\coloneqq \psi}
\wedge \Box (\neg p \rightarrow{} \Box \neg p) \]
\begin{prooftree}
	\AxiomC{$ \Box \neg \Box \neg p,
	\Box \phi,
	\neg \Box \neg p,
	\cm{\neg \Box p,}
	\Box \psi,
\Box \neg ( \neg p \wedge \neg \Box \neg p) $}
\anm{$(\neg \Box)$}
\UnaryInfC{$\neg \Box \neg p,
	\phi,
	\neg p,
	\psi,
\cm{\neg ( \neg p \wedge \neg \Box \neg p)} $}
\anm{$(\neg \wedge)$}

\UnaryInfC{$
	\neg \Box \neg p, \phi, \cm{\neg p}, \psi, \cm{\neg \neg p} \quad | \quad
\cm{\neg \Box \neg p}, \phi, \neg p, \psi, \cm{\neg \neg \Box \neg p} $}
\anm{$(clash)$}
		\UnaryInfC{$\bot$}
\end{prooftree}

Im letzten Schritt lässt sich auf beide möglichen Resultate der Anwendung der
$(\neg \wedge)$ Regel sofort die $(clash)$-Regel anwenden.

Da für jede Regelanwendung eine erfolgreiche Konklusion existieren muss
damit ein Label erfolgreich ist, ist dieser Label nicht erfolgreich.

%%2. Formel TODO:
Durch umformen der zweiten gegebenen Formel erhält man folgende einfache Formel:

\begin{center}
	$ \Box \neg(\neg p \wedge (\Box q \wedge \Box r)) \wedge \neg (\Box p \wedge
	\Box\neg p) \wedge \Box \neg (\neg p \wedge \Box\neg p) \wedge \Box\Box(p
	\wedge \neg r)$
\end{center}

Ab jetzt lassen sich die Tableau-Regeln anwenden.




\aufgabe{2}{Große Modelle}
\angabe{%
Gegeben seien Atome $q_1,q_2,\dots$. Man definiere die Familie von Formeln
$\phi_n$ durch
\begin{equation*}
\phi_n = \bigwedge^{n}_{i=0} \Box^i(\dia q_{i+1} \wedge \dia \neg
q_{i+1}) \wedge \displaystyle \bigwedge_{i=1}^{n} \Box^i(\Box^{\leq n - i}((q_i
\rightarrow \Box q_i) \wedge (\neg q_i \rightarrow{} \Box \neg q_i))).
\label{bigmodel}
\end{equation*}
Zeigen Sie, dass jedes Modell von $\phi_n$ mindestens $2^n$ Elemente hat. Geben
Sie eine polynomielle Abschätzung für die Größe von $\phi_n$ in Abhängigkeit
von $n$ an.
}

Sei $\kripke{M}$ ein Modell für $\phi_n$, $x\in M$ mit $\kripke{M}, x \vDash
\phi_n$.

Die erste Konjunktion in $\phi_n$ stellt sicher, dass jeder Punkt des
Modells, der von $x$ in maximal $n$ Schritten erreichbar ist, mindestens zwei
verschiedene Nachfolger besitzt.

Durch die zweite Konjunktion wird ein Atom $q_i$ (bzw. seine Negation $\neg q_i$)
von einem Punkt $y$ der von $x$ ausgehend in mindestens $i$ und maximal $n$
Schritten erreichbar ist auf alle seine Nachfolger propagiert; somit folgt,
dass von $y$ ab alle erreichbaren Punkte $q_i$ (bzw. $\neg q_i$) erfüllen,
solange sie nicht weiter als $n+1$ Schritte von $x$ entfernt sind.

Dies bedeutet, dass von $x$ aus im $i$-ten Schritt entweder $q_i$ oder $\neg
q_i$ gewählt werden kann und diese Wahl sich bis zum Schritt $n$ \glqq
gemerkt\grqq wird; damit folgt, dass im $n$-ten Schritt für die $n$ Atome
jeweils eine Wahl getroffen wurde, sodass diese Kombinationen jeweils paarweise
verschieden sind (und nicht gleichzeitig an einem Punkt in einem Kripkemodell
gelten können). Die Anzahl dieser Kombinationen ist $2^n$. Damit folgt, dass
jedes Modell von $\phi_n$ mindestens $2^n$ Elemente hat.

Abschätzung der Größe von $\phi_n$:
\[
\sum_{i=0}^{n}(i + 4) + n + 1 + \sum_{i=1}^{n}\left(i + \left(\sum_{j=0}^{n-i}
j + 7\right) + (n - i + 1) \right) + n = \frac{5+(65 n)}{6}+5 n^2+\frac{n^3}{6}
\in \mathcal{O}(n^3)
\]
\aufgabe{3}{Keine großen Modelle}
\angabe{$S5$ bezeichne die Modallogik der Kripkerahmen, deren Zustandsübergangsrelation transitiv, reflexiv
	und symmetrisch ist. Zeigen Sie, dass die Methode von Aufgabe~2 zur Widerlegung der polynomiellen
	Modelleigenschaft fur $S5$ nicht anwendbar ist, weil die Formeln $\phi_n$ in solchen Rahmen nicht erfüllbar sind.}

Sei $\kripke{M}$ ein Modell über S5 und $x \in M$ ein Punkt, der $\phi_1$ erfüllt. Aufgrund der ersten Konjunktion in
$\phi_1$ für $i = 0$ muss es $y, z \in M$ geben mit $xRy$ und $xRz$, wobei o.E. $\kripke{M}, y \vDash p_1$ sowie
$\kripke{M}, z \vDash \neg p_1$. Aus der Symmetrie der Relation $R$ folgt jedoch $yRx$ und wegen Transitivität auch
$yRz$. Die zweite Konjunktion fordert jedoch für $i = 1$, dass jeder Nachfolger von $y$ (somit auch $z$) $p_1$ erfüllen
muss. Dies führt zu einem Widerspruch, $\kripke{M}, x \not\vDash \phi_1$.

Die oben geschilderten Bedingungen müssen auch für alle $n$ mit $n > 1$ gelten. Folglich ist kein $\phi_n$, $n > 1$ über
S5 erfüllbar.

Eine Ausnahme bildet $\phi_0$, da hier die zweite Konjunktion leer ist.

\aufgabe{4}{Gegenseitige Simulation}
\angabe{Zeigen Sie, dass zwischen den Wurzeln der beiden Modelle in beiden Richtungen Simulationen existieren,
aber keine Bisiumulation.}

\begin{center}
\begin{minipage}{0.4\textwidth}

\begin{tikzpicture}[->,>=stealth',shorten >=5pt, shorten <=5pt,auto,node distance=1.5cm,
  main node/.style={circle,inner sep=2pt,minimum size=1pt,fill,=black,draw,font=\sffamily\Large\bfseries},
  bnode/.style={circle,inner sep=2pt,minimum size=1pt,fill=black,draw,font=\sffamily\Large\bfseries,label=east:B},
  cnode/.style={circle,inner sep=2pt,minimum size=1pt,fill=black,draw,font=\sffamily\Large\bfseries,label=east:C},
  invnode/.style={circle,inner sep=2pt,minimum size=1pt,draw=white,font=\sffamily\Large\bfseries}
  ]

  \node[main node,label=west:$x_1$] (1) {};
  \node[invnode] (3) [below of=1] {};
  \node[main node,label=west:$x_{21}$] (2) [left of=3] {};
  \node[main node,label=west:$x_{22}$] (4) [right of=3] {};
  \node[bnode,label=west:$x_{31}$] (5) [below of=2] {};
  \node[cnode,label=west:$x_{32}$] (6) [below of=3] {};
  \node[cnode,label=west:$x_{33}$] (7) [below of=4] {};

  % fuer einzelne pfeile, die unterschiedlich sind: edge [<->] node (zb)
  \path[every node/.style={font=\sffamily\small}]
	(1) edge node [right] {} (2)
		edge node [right] {} (4)
	(2) edge [<->] node [left] {} (5)
		edge [out=-20,in=110] node [right] {} (6)
	(4) edge [<->] node [left] {} (7)
	(6) edge [out=150,in=-60] node [left] {} (2);
\end{tikzpicture}
\end{minipage}
\begin{minipage}{0.4\textwidth}
\begin{tikzpicture}[->,>=stealth',shorten >=5pt, shorten <=5pt,auto,node distance=1.5cm,
  main node/.style={circle,inner sep=2pt,minimum size=1pt,fill,=black,draw,font=\sffamily\Large\bfseries},
  bnode/.style={circle,inner sep=2pt,minimum size=1pt,fill=black,draw,font=\sffamily\Large\bfseries,label=east:B},
  cnode/.style={circle,inner sep=2pt,minimum size=1pt,fill=black,draw,font=\sffamily\Large\bfseries,label=east:C},
  invnode/.style={circle,inner sep=2pt,minimum size=1pt,draw=white,font=\sffamily\Large\bfseries}
  ]


  \node[main node,label=west:$y_1$] (1) {};
  \node[invnode] (3) [below of=1] {};
  \node[main node,label=west:$y_{21}$] (2) [left of=3] {};
  \node[main node,label=west:$y_{22}$] (4) [right of=3] {};
  \node[bnode,label=west:$y_{31}$] (5) [below of=2] {};
  \node[cnode,label=west:$y_{32}$] (6) [below of=3] {};
  \node[bnode,label=west:$y_{33}$] (7) [below of=4] {};

  \path[every node/.style={font=\sffamily\small}]
	(1) edge node [right] {} (2)
		edge node [right] {} (4)
	(2) edge [<->] node [left] {} (5)
		edge [out=-20,in=110] node [right] {} (6)
	(4) edge [<->] node [left] {} (7)
	(6) edge [out=150,in=-60] node [left] {} (2);
\end{tikzpicture}
\end{minipage}
\end{center}

Für die Existenz von Simulationen sowie die Nichtexistenz von Bisimulation reicht es,
lediglich Beispiele anzugeben.

\begin{enumerate}
	\item Simulation von links nach rechts: $S = \{ (x_1,y_1), (x_{21}, y_{21}), (x_{22}, y_{21}), (x_{31}, y_{31}), (x_{32},  y_{32}), (x_{33}, y_{32}) \}$
		Es ist leicht nachzuprüfen, dass dies eine Simulation ist.
	\item Simulation von rechts nach links: $S = \{ (y_1,x_1), (y_{21}, x_{21}), (y_{22}, x_{21}), (y_{31}, x_{31}), (y_{32},  x_{32}), (y_{33}, x_{31}) \}$
		Wiederum ist leicht zu sehen, dass dies eine Simulation ist.
	\item Bisimulation: Die Formel $\Box \dia B$ wird an der Wurzel im rechten Modell erfüllt, nicht jedoch im linken Modell.
\end{enumerate}

%Gegeben sind die beiden baumförmigen Kripkemodelle \kripke{M} (links) und
%\kripke{N} (rechts).

%$\mathrm{Z\kern-0.4em\raise-0.4ex\hbox{Z}}$: Zwischen den Wurzeln ($\{x,y\}_1$)
%der beiden Modelle existiert in beiden Richtungen Simulationen, aber keine
%Bisimulation.

%Diese simulieren sich gegenseitig, wenn es sowohl eine Relation $S \subseteq M \times
%N$ und eine Relation $R \subseteq N \times M, (S \neq R)$ gibt und beide Relationen
%Simulationen sind.

%Zunächst zeigen wir, dass eine solche Relation $S(R)$ nur existieren kann, wenn
%$x_1 S(R) y_1$ gilt. Gilt nämlich oBdA. $x_1 S(R) y_i, i \neq 1$ dann folgt durch
%Betrachtung der Nachfolger, wie nach Definitionsteil (ii), dass auf der rechten
%Seite nicht ausrechend Nachfolger vorhanden sind.

%Beweisführung durch Überprüfung der Definition von Simulation:

%Nicht Modal:

%Es gilt: $\kripke{M}, x_1 \vDash P$ und $\kripke{N},y_1 \vDash P$ (i). Für alle
%Nachfolger von $x_1$, $x_2$, muss nun gelten: $\exists w.y_1 R w \wedge x_2 S w$
%(ii).
%\begin{enumerate}
	%\item $w = y_{21}$ erfüllt diese Bedingung, da in beiden Punkten $P$ gilt.
		%Überprüfe nun erneut (ii) bei dem Nachfolger $x_{32}$ von $x_2$. Nach
		%Definition muss wiederum gelten: $\exists w'.y_{21} R w' \wedge x_{32} S
		%w'$. Im Modell \kripke{N} kann nur $w' = y_{31}$ gelten. Damit ist $S$
		%keine Simulation, da $\kripke{M}, x_{32} \nvDash P$ und $\kripke{N},
		%y_{32} \vDash P$
	%\item Für $w = y_{22}$ ist der Beweis analog bei Wahl Nachfolger $x_{31}$
		%von $x_2$.
%\end{enumerate}

%Im modalen Fall erfolgt der Beweis analog mithilfe des
%Bisimulationsinvarianzsatzes, wenn $P$ eine modale Formel ist.

