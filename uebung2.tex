\aufgabe{1}{Atomare Axiomenregel}
\newcommand{\Tk}{T_\text{komplex}}
\newcommand{\Ta}{T_\text{atomar}}

\begin{center}
\colorbox{Golden}{\begin{minipage}{0.95\textwidth}%
Zeigen Sie, dass statt der in der Vorlesung angegebenen Axiomenregel des aussagenlogischen Tableaukalküls
$(\phi,\neg\phi,\Gamma/\bot)$ die \emph{atomare Axiomenregel}
\[
  \frac{A,\neg A,\Gamma}{\bot}
\]
ausreicht. Formal bedeutet dies, dass man durch Induktion über die Struktur von $\phi$ herleitet, dass
$ \phi,\neg\phi,\Gamma$ im eingeschränkten Kalkül keinen erfolgreichen Zweig hat.
\end{minipage}}
\end{center}
\vspace{0.5cm}

Bezeichne $\Tk$ den ursprünglichen Kalkül mit komplexer Axiomenregel $(\phi, \neg \phi, \Gamma / \bot)$ und $\Ta$ den
neuen Kalkül mit atomarer Axiomenregel und sei $\#\phi$ die Anzahl der Konnektive der Formel $\phi$.

N.B.: Außer der Axiomenregel gibt es keine Regel, deren Antezedens mehr als ein Element eines Labels enthält; dies
bedeutet, dass Regelanwendungen auf Elemente von $\Gamma$ nicht betrachtet werden müssen.

Zu zeigen ist, dass $\Ta$ genau dann ein Label $\Gamma$ ablehnt, wenn $\Gamma$ von $\Tk$ abgelehnt wird:
\begin{itemize}
    \item[\glqq$\Rightarrow$\grqq] Lässt sich die atomare Axiomenregel anwenden, so auch die komplexe Axiomenregel.
    \item[\glqq$\Leftarrow$\grqq] Induktion über die Struktur von $\phi$:
        \paragraph{I.A.} $\#\phi = 0 \Rightarrow \phi = A, A$ Literal $\Rightarrow$ $\Tk$ und $\Ta$ lehnen den
        Label $\phi, \neg \phi, \Gamma$ ab
        \paragraph{I.V.} $\forall \phi, \ \#\phi \leq n$ gelte bereits:
        \[ \Tk \text{ lehnt } \phi, \neg \phi, \Gamma \text{ ab } \text{ und } \Ta \text{ lehnt }
        \phi, \neg \phi, \Gamma \text{ ab } \]
        \paragraph{I.S.} Seien $\phi_1, \phi_2$ zwei Formeln mit $\#\phi_1 = k_1 < n, \#\phi_2 = k_2 < n$.
        \begin{itemize}
            \item[$i)$] $\phi = \phi_1 \wedge \phi_2$ und $\Tk$ lehnt den Label $\phi, \neg \phi, \Gamma$
                ab. Folgende Zweige können durch Regelanwendungen im Kalkül $\Ta$ entstehen:

                \begin{minipage}{0.4\textwidth}
                    \begin{prooftree}
                        \AxiomC{$\phi_1 \wedge \phi_2, \neg (\phi_1 \wedge \phi_2), \Gamma$}
                        \UnaryInfC{$\phi_1, \phi_2, \neg (\phi_1 \wedge \phi_2), \Gamma$}
                        \UnaryInfC{$\phi_1, \phi_2, \neg \phi_1, \Gamma$}
                        \RightLabel{\scriptsize{nach I.V.}}
                        \UnaryInfC{$\bot$}
                    \end{prooftree}
                \end{minipage}\hfill
                \begin{minipage}{0.4\textwidth}
                    \begin{prooftree}
                        \AxiomC{$\phi_1 \wedge \phi_2, \neg (\phi_1 \wedge \phi_2), \Gamma$}
                        \UnaryInfC{$\phi_1, \phi_2, \neg (\phi_1 \wedge \phi_2), \Gamma$}
                        \UnaryInfC{$\phi_1, \phi_2, \neg \phi_2, \Gamma$}
                        \RightLabel{\scriptsize{nach I.V.}}
                        \UnaryInfC{$\bot$}
                    \end{prooftree}
                \end{minipage}

                In beiden Fällen wird der Label auch von $\Ta$ abgelehnt.

                \item[$ii)$] $\phi = \neg\phi_1$ und $\Tk$ lehnt den Label $\neg\phi_1, \neg\neg\phi_1, \Gamma$
                    ab. Im Kalkül $\Ta$ erhält man:
                    \begin{prooftree}
                        \AxiomC{$\neg\phi_1, \neg\neg\phi_1, \Gamma$}
                        \RightLabel{\scriptsize{$(\neg\neg)$}}
                        \UnaryInfC{$\neg\phi_1, \phi_1, \Gamma$}
                        \UnaryInfC{$\bot$}
                    \end{prooftree}

                    Der Label wird also auch von $\Ta$ abgelehnt.
        \end{itemize}
        Da es nur die Konnektive $\wedge$ und $\neg$ gibt, folgt die Behauptung.
\end{itemize}

\newpage
\aufgabe{2}{Rekursiver Tableaualgorithmus}
\begin{center}
\colorbox{Golden}{\begin{minipage}{0.95\textwidth}%
    Man definiere einen Label $\Gamma$ als \emph{erfolgreich}, wenn jede darauf anwendbare Tableauregel mindestens eine
    erfolgreiche Konklusion hat (warum ist das eine Definition?). Zeigen Sie durch Induktion über die Baumhöhe, dass ein
    Label genau dann erfolgreich ist, wenn er einen erfolgreichen Zweig hat.
\end{minipage}}
\end{center}
\vspace{0.5cm}

Da der Label $\Gamma$ endlich viele Konnektive hat und jede Regel die Anzahl der Konnektive verringert, führen alle
Kombinationen von Regelanwendungen in endlicher Zeit zur Terminierung. Damit ist der Begriff eines \emph{erfolgreichen}
Labels wohldefiniert.

Der Begriff \emph{erfolgreich} wird also verallgemeinert, sodass ein Baum von Regelanwendungen entsteht, bei dem die
Pfade von der Wurzel zu jedem Blatt Folgen von Regelanwendungen repräsentieren, deren Anwendungen auf $\Gamma$ zu einer
erfolgreichen Terminierung des Tableaualgorithmus führen.

Zu zeigen: Label erfolgreich $ \ \Leftrightarrow \ \exists $ erfolgreicher Zweig

Beweis:
\begin{itemize}
    \item [\glqq$\Rightarrow$\grqq] folgt direkt aus der Definition.
    \item [\glqq$\Leftarrow$\grqq] Label $\Gamma$ habe im Tableaualgorithmus einen erfolgreichen Zweig.

        Induktion über die Baumhöhe:
        \paragraph{I.A.} Höhe 0: $\Gamma = A_1, \dots, A_m$, wobei $A_i$ verschiedene Literale (negativ oder positiv)
        sind, d.h. es ist keine weitere Regel anwendbar $\Rightarrow \ \Gamma$ erfolgreich (\emph{vacuously}).
        \paragraph{I.V.} Für alle Label $\Gamma$ der Höhe $k < n$ mit erfolgreichem Zweig gelte die Behauptung.
        \paragraph{I.S.} Höhe $n$.
        \begin{itemize}
            \item[$(\wedge)$] $\Gamma = \varphi \wedge \psi, \Xi \ \vdash \ \phi, \psi, \Xi$, Label mit Höhe $< n$

                Nach I.V. gilt: $\varphi, \psi, \Xi$ ist erfolgreicher Label. Somit führt auch die Regelanwendung bei
                $\Gamma$ zu einer erfolgreichen Konklusion.
            \item[$(\neg\neg)$] $\Gamma = \neg \neg \varphi, \Xi \ \vdash \ \varphi, \Xi$, Label mit Höhe $< n$

                Nach I.V. gilt: $\varphi, \Xi$ ist erfolgreicher Label. Somit führt auch
                die Regelanwendung bei $\Gamma$ zu einer erfolgreichen Konklusion.
            \item[$(\neg\wedge)$] $\Gamma = \neg (\varphi \wedge \psi), \Xi \ \vdash \ \neg \varphi, \Xi \ | \ \neg
                \psi, \Xi$, jeweils Label mit Höhe $< n$

                O.B.d.A. ist $\neg \varphi, \Xi$ Knoten des erfolgreichen Zweiges von $\Gamma$. Nach I.V. gilt: $\neg
                \varphi, \Xi$ ist erfolgreicher Label. Somit führt auch diese Regelanwendung bei $\Gamma$ zu mindestens
                einer erfolgreichen Konklusion.
        \end{itemize}
        Somit hat $\Gamma$ für jede mögliche Regelanwendung eine erfolgreiche Konklusion.
\end{itemize}
Damit ist $\Gamma$ ein erfolgreicher Label.

\newpage
\aufgabe{3}{Reduktion von BDDs}
\begin{center}
\colorbox{Golden}{\begin{minipage}{0.95\textwidth}%
    Geben Sie ein baumförmiges BDD für die Formel $(x \vee y) \wedge (z \vee w)$ an und reduzieren Sie es, so weit
    möglich.
\end{minipage}}
\end{center}
\vspace{0.5cm}

Die Formel $(x \vee y) \wedge (z \vee w)$ kann dargestellt werden als baumförmiges BDD:

\begin{center}
\begin{tikzpicture}[
    nterm/.style={circle,draw,thick,fill=white,anchor=north,minimum size=6mm,drop shadow},
    term/.style={rectangle,draw,fill=white,thick,anchor=north,minimum size=6mm,drop shadow},
    level distance=0.76cm, growth parent anchor=south, sibling distance=5cm
    ]
    \node (x) [nterm] {$x$} [-latex]
        child{ [sibling distance=1.5cm] node (y) [nterm] {$y$}
        child{ node (a0) [term] {$0$} edge from parent node[left,scale=.75] {$0$} }
            child{ node (z1) [nterm,fill=green!50] {$z$}
                    child{ node (w1) [nterm,fill=green!50] {$w$}
                        child{ node (b0) [term,fill=green!50] {$0$} edge from parent node[left,scale=.75] {$0$} }
                        child{ node (a1) [term,fill=green!50] {$1$} edge from parent node[right,scale=.75] {$1$} }
                    edge from parent node[left,scale=.75] {$0$} }
                    child { node (b1) [term,fill=green!50] {$1$} edge from parent node[right,scale=.75] {$1$} }
            edge from parent node[right,scale=.75] {$1$} }
        edge from parent node[above left,scale=.75] {$0$} }
        child{ [sibling distance=1.5cm] node (z2) [nterm,fill=green!50] {$z$}
            child{ node (w2) [nterm,fill=green!50] {$w$}
                    child{ node[fill=green!50] (c0) [term] {$0$} edge from parent node[left,scale=.75] {$0$} }
                    child{ node[fill=green!50] (c1) [term] {$1$} edge from parent node[right,scale=.75] {$1$} }
                   edge from parent node[left,scale=.75] {$0$} }
            child{ node (d1) [term,fill=green!50] {$1$} edge from parent node[right,scale=.75] {$1$} }
    edge from parent node[above right,scale=.75] {$1$} };
\end{tikzpicture}
\end{center}

Die farbig markierten Untergraphen der beiden $z$-Knoten sind isomorph. Somit kann einer der $z$-Knoten entfernt werden,
wobei die eingehende Kante auf den anderen $z$-Knoten umgebogen wird. Dadurch entsteht der links abgebildete Graph:
\vspace{0.5cm}

\begin{minipage}{0.4\textwidth}
\begin{center}
\begin{tikzpicture}[
        p/.style={scale=.75},
        nterm/.style={circle,draw,thick,fill=white,anchor=north,minimum size=6mm,drop shadow},
        term/.style={rectangle,draw,fill=white,thick,anchor=north,minimum size=6mm,drop shadow},
        level distance=0.76cm, growth parent anchor=south, sibling distance=1.5cm
        ]
        \node (x) [nterm] {$x$};
        \node [nterm,below left=of x] (y) {$y$};
        \node [nterm,below right=of x] (z) {$z$};
        \node [term,below left=of y,fill=red!50] (a0) {$0$};
        \node [nterm,below left=of z] (w) {$w$};
        \node [term,below right=of z,fill=blue!50] (a1) {$1$};
        \node [term,below left=of w,fill=red!50] (b0) {$0$};
        \node [term,below right=of w,fill=blue!50] (b1) {$1$};
        \path [draw,-latex] (x) -- node[p,above left] {$0$} (y);
        \path [draw,-latex] (x) -- node[p,above right] {$1$} (z);
        \path [draw,-latex] (y) -- node[p,above left] {$0$} (a0);
        \path [draw,-latex] (y) -- node[p,above] {$1$} (z);
        \path [draw,-latex] (z) -- node[p,above left] {$0$} (w);
        \path [draw,-latex] (z) -- node[p,above right] {$1$} (a1);
        \path [draw,-latex] (w) -- node[p,above left] {$0$}  (b0);
        \path [draw,-latex] (w) -- node[p,above right] {$1$}  (b1);
\end{tikzpicture}
\end{center}
\end{minipage}\hfill\begin{minipage}{0.4\textwidth}
\begin{center}
\begin{tikzpicture}[
        p/.style={scale=.75},
        nterm/.style={circle,draw,thick,fill=white,anchor=north,minimum size=6mm,drop shadow},
        term/.style={rectangle,draw,fill=white,thick,anchor=north,minimum size=6mm,drop shadow},
        level distance=0.76cm, growth parent anchor=south, sibling distance=2cm
        ]
        \node (x) [nterm] {$x$};
        \node [nterm,below left=of x] (y) {$y$};
        \node [nterm,below right=of x] (z) {$z$};
        \node [nterm,below left=of z] (w) {$w$};
        \node [term,below left=of w,fill=red!50] (b0) {$0$};
        \node [term,below right=of w,fill=blue!50] (b1) {$1$};
        \path [draw,-latex] (x) -- node[p,above left] {$0$} (y);
        \path [draw,-latex] (x) -- node[p,above right] {$1$} (z);
        \path [draw,-latex] (y) -- node[p,left] {$0$} (b0);
        \path [draw,-latex] (y) -- node[p,above] {$1$} (z);
        \path [draw,-latex] (z) -- node[p,above left] {$0$} (w);
        \path [draw,-latex] (z) -- node[p,right] {$1$} (b1);
        \path [draw,-latex] (w) -- node[p,above left] {$0$}  (b0);
        \path [draw,-latex] (w) -- node[p,above right] {$1$}  (b1);
\end{tikzpicture}
\end{center}
\end{minipage}
\vspace{0.5cm}

Nach Entfernen der doppelten terminalen Knoten erhält man den rechten Graphen, das vollständig reduzierte BDD.

\aufgabe{4}{Erfüllbarkeit von BDDs}
\begin{center}
\colorbox{Golden}{\begin{minipage}{0.95\textwidth}%
    Geben Sie einen Algorithmus an, der ein (nicht notwendig geordnetes) BDD auf Erfüllbarkeit überprüft, und
    analysieren Sie Laufzeit und Platzverbrauch.
\end{minipage}}
\end{center}
\vspace{0.5cm}

Ein BDD ist erfüllbar, wenn es einen Pfad von der Wurzel $r$ zu einem Terminal mit Label $1$ gibt. Die Existenz eines
terminalen Knotens mit Label $1$ ist sogar hinreichend:

Gäbe es einen terminalen Knoten $1$ ohne, dass ein Pfad von der Wurzel zu ihm existiert, so gäbe es außer der Wurzel
mindestens noch einen initialen Knoten, was im Widerspruch zur Definition eines BDD steht.

$\Rightarrow$ Algorithmus: Durchsuchen der Knotenmenge nach terminalem Knoten mit Label $1$. Die Laufzeit ist linear in
der Länge der Eingabe, der Algorithmus arbeitet ohne Platzoverhead.

\aufgabe{5}{Verstärkung}
\begin{center}
\colorbox{Golden}{\begin{minipage}{0.95\textwidth}%
    Man ändere die Formulierung des Tableaualgorithmus gemäß Aufgabe 2 ab und füge dann die Verstärkungsregel
    (\emph{strengthening}) $\Delta,\Gamma/\Gamma$ hinzu. Zeigen Sie, dass dies die Antworten des Algorithmus nicht
    ändert, d.h., dass im erweiterten System genau dann \glqq erfüllbar\grqq\ geantwortet wird, wenn dies im
    ursprünglichen System der Fall war.
\end{minipage}}
\end{center}
\vspace{0.5cm}

Zu zeigen ist lediglich, dass, wenn der Algorithmus der Algorithmus ohne Verstärkungsregel ein Label akzeptiert, dann
auch der Algorithmus mit Verstärkungsregel (Induktion 2). Hierzu muss zunächst gezeigt werden, dass der ursprüngliche Algorithmus
$\Gamma$ akzeptiert, wenn er $\Delta, \Gamma$ akzeptiert (Induktion 1).

\subsection*{Induktion 1 (über die Anzahl der noch anwendbaren Regeln):}
\paragraph{I.A.} Label $\Delta, A$ (wobei $A$ Literal) und Label $A$ wird offensichtlich akzeptiert.
\paragraph{I.V.} Label $\Delta, \Gamma_k$ und $\Gamma_k$ werden akzeptiert, wobei auf $\Gamma_k$ noch $k < n$ für ein $n
\in \mathbb{N}$ Regeln anwendbar sind.
\paragraph{I.S.} Label $\Delta, \Gamma_n$. Regelanwendungen auf $\Delta$ verändern nicht $\Gamma_n$ und liefern
lediglich $\Delta^\prime, \Gamma_n$. Da $\Delta$ endlich groß ist, muss irgendwann eine Regel auf $\Gamma_n$ angewendet
werden, man erhält $\Gamma_{n-1}$, die Anzahl der noch anwendbaren Regeln ist $< n$. Somit folgt nach
Induktionsvoraussetzung, dass $\Gamma_n$ akzeptiert wird.
\vspace{0.6cm}

\subsection*{Induktion 2 (über die Anzahl der noch anwendbaren Regeln):}
\paragraph{I.A.} keine Regeln im ursprünglichen Algorithmus anwendbar, d.h. $\Xi_0 = \Delta, \Gamma = A_1, \cdots,
A_m$, wobei $A_i$ verschiedene Literale (negativ oder positiv) sind. Bei Anwendung von \emph{strengthening} erhält man
$\Gamma = A_j, \cdots, A_m, 0 < j < m$. Induktiv folgt, dass der neue Algorithmus ebenfalls $\Xi_0$ akzeptiert.

\paragraph{I.V.} Seien im ursprünglichen Algorithmus noch $k < n$ für ein $n \in \mathbb{N}$ Regeln auf einen Label
$\Xi_k$ anwendbar. Beide Algorithmen akzeptieren $\Xi_k$.

\paragraph{I.S.} $\Xi_n$ sei ein Label, der vom ursprünglichen Algorithmus akzeptiert wird und auf den $n$ Regeln
anwendbar sind. Dieser werde nun im neuen Algorithmus betrachtet. Wird eine Regel aus dem ursprünglichen Algorithmus
angewendet, so entsteht nach Definition von Aufgabe 2 ein Label, der vom ursprünglichen Algorithmus akzeptiert wird und
auf den nur noch $n-1$ Regeln anwendbar sind. Nach Induktionsvoraussetzung wird dieser auch vom neuen Algorithmus
akzeptiert. Anwendung von \emph{strengthening} auf $\Xi_n = \Delta, \Gamma$: dann gilt nach Induktion 1, dass $\Gamma$
vom ursprünglichen Algorithmus akzeptiert wird und auf $\Gamma$ sind noch $< n$ Regeln anwendbar also folgt nach
Induktionsvoraussetzung, dass $\Xi_n$ vom neuen Algorithmus akzeptiert wird.

Damit folgt die Behauptung.
