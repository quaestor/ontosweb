% vim: set filetype=tex:

\aufgabe{1}{Simulation verkehrt}
\angabe{%
	Man definiere eine Rückwärtssimulation zwischen Modellen $\kripke{M}_1 =
	(M_1, R_1, V_1)$ und $\kripke{M}_2 = (M_2, R_2, V_2)$ als eine binäre
	Relation $S \subseteq M_1 \times M_2$, so dass für alle $x, y$ mit $xSy$
	gilt
	\begin{enumerate}
		\item für jedes Atom $p$ mit $x \in V_1(p)$ gilt $y \in V_2(p)$
		\item für jedes $y'$ mit $yR_2y'$ existiert $x'$ mit $xR_1x'$ und
			$x'Sy'$.
	\end{enumerate}
	Zeigen Sie, dass, wenn $S$ eine Rückwärtssimulation ist und $\phi$ eine
	Formel in der durch die Grammatik
	\begin{center}
		$ \phi \coloncolonequals p~|~\top~|~\phi_1 \wedge \phi_2~|~\phi_1 \vee
	\phi_2~|~\Box \phi $
	\end{center}
	definierten Menge modaler Formeln, dann $S$ Erfülltheit von $\phi$ bewahrt,
	d.h. aus $xSy$ und $\kripke{M}_1, x \vDash \phi$ folgt $\kripke{M}_2, y
	\vDash \phi$.
}

Die Behauptung lässt sich per Induktion über die Struktur von $\phi$ zeigen:
Für die atomaren Fälle $\phi = p$ und $\phi = \top$ ist dies trivialerweise erfüllt (I.A.).
\begin{itemize}
	\item[$\phi = \phi_1 \wedge \phi_2$] I.V.: Sei die Behauptung bereits für $\phi_1$ und $\phi_2$ gültig.
		Dann folgt aus $\kripke{M}_1, x \vDash \phi$, dass $\kripke{M}_1, x \vDash \phi_1$ und $\kripke{M}_1, x \vDash \phi_2$. Nach I.V. gilt dann
		wegen $xSy$ und (1.), dass auch $\kripke{M}_2, y \vDash \phi_1$ und $\kripke{M}_2, y \vDash \phi_2$ gilt und somit
		$\kripke{M}_2, y \vDash \phi$. 
	\item[$\phi = \phi_1 \vee \phi_2 $] Analog zu obigem Fall.
	\item[$\phi = \Box \phi'$] Sei die Behauptung bereits für $\phi'$ gültig (I.V.).
		Hat $x$ keine Nachfolger ($\Box \phi'$ ist damit trivialerweise erfüllt), so darf auch $y$ keine Nachfolger haben,
		denn wenn $y$ einen Nachfolger hätte, gilt nach (2.), dass für jeden dieser Nachfolger $y'$ ein $x'$ existieren muss mit $xR_1x'$. Somit erfüllt auch
		$\kripke{M}_2, y$ trivialerweise $\Box \phi'$.
		Hat nun $x$ Nachfolger, dann gilt für alle $y'$, Nachfolger von $y$ ($yR_2y'$), dass ein $x'$ existiert mit $xR_1x'$ und $x'Sy'$ (2.). Da dieses $x'$ Nachfolger von
		$x$ ist gilt dort nach Voraussetzung $\kripke{M}_1, x' \vDash \phi'$; da $x'Sy'$ gilt, folgt nach I.V. dass auch für jedes $y'$ gilt: $\kripke{M}_2, y' \vDash \phi'$.
		Somit folgt dass $\kripke{M}_2, y \vDash \Box \phi'$ gilt.
\end{itemize}
Somit gilt die Behauptung.




\newcommand{\ex}{\exists R.}
\newcommand{\and}{\sqcap}

\aufgabe{2}{Subsumption in $\el$}
\angabe{%
	Wenden Sie den Subsumptionsalgorithmus für $\el$ an, um zu entscheiden, ob
	folgende Subsumptionen gültig sind:
\begin{equation}
	A \and B \and \ex (B \and C) \and \ex ( B \and C
	\and \ex D)
	\sqsubseteq
	A \and \ex (B \and \ex D) \and \ex ( C \and \ex D)
\end{equation}

\begin{equation}
	A \and \ex ( B \and \ex C) \and \ex (B \and \ex D)
	\sqsubseteq
	A \and \ex (B \and \ex (C \and D))
\end{equation}
}

Da jede Kante im Beschreibungsbaum die Relation $R$ darstellt, werden
entsprechende Kantenlabel der Einfachkeit halber weggelassen. \\
Um zu zeigen, dass diese Subsumptionen gültig sind, reicht es zu zeigen, dass es
eine Simulation gibt, die jeweils die Punkte $y_1$ und $x_1$ der zugehörigen
Beschreibungsbäume in Relation stellt (im Folgenden sei für $C \sqsubseteq D$
jeweils der linke Baum das Modell $\mathcal{I}_C$, der rechte $\mathcal{I}_D$).

\begin{center}
\begin{minipage}{0.4\textwidth}
\begin{tikzpicture}[->,>=stealth',shorten >=5pt, shorten <=5pt,auto,node distance=1.5cm,
  main node/.style={circle,inner sep=2pt,minimum size=1pt,fill,=black,draw,font=\sffamily\Large\bfseries},
  not main node/.style={circle,inner sep=2pt,minimum size=1pt,fill=black,draw,font=\sffamily\Large\bfseries}]

  \node[main node,label=west:$x_1$,label=east:{$A$}] (1) {};
  \node[main node,label=west:$x_{21}$,label=east:{$B,C$}] (2) [below left of=1] {};
  \node[main node,label=west:$x_{22}$,label=east:{$B,C$}] (4) [below right of=1] {};
  \node[not main node,label=west:$x_{32}$,label=east:{$D$}] (5) [below of=4] {};

  \path[every node/.style={font=\sffamily\small}]
    (1) edge node [right] {} (2)
		edge node [left]  {} (4)
	(4) edge node [right] {} (5);
\end{tikzpicture}
\end{minipage}
\begin{minipage}{0.4\textwidth}
\begin{tikzpicture}[->,>=stealth',shorten >=5pt, shorten <=5pt,auto,node distance=1.5cm,
  main node/.style={circle,inner sep=2pt,minimum
  size=1pt,fill,=black,draw,font=\sffamily\Large\bfseries},
  not main node/.style={circle,inner sep=2pt,minimum size=1pt,fill=black,draw,font=\sffamily\Large\bfseries}]

  \node[main node,label=west:$y_1$,label=east:{$A$}] (1) {};
  \node[main node,label=west:$y_{21}$,label=east:{$B$}] (2) [below left of=1] {};
  \node[main node,label=west:$y_{31}$,label=east:{$D$}] (3) [below of=2] {};
  \node[main node,label=west:$y_{22}$,label=east:{$C$}] (4) [below right of=1] {};
  \node[not main node,label=west:$y_{32}$,label=east:{$D$}] (5) [below of=4] {};

  \path[every node/.style={font=\sffamily\small}]
    (1) edge node [right] {} (2)
		edge node [left]  {} (4)
	(2) edge node [right] {} (3)
	(4) edge node [right] {} (5);
\end{tikzpicture}
\end{minipage}
\end{center}

Die geforderte Simulation ist $S = \{(y_1, x_1), (y_{21}, x_{22}), (y_{31},
x_{32}), (y_{22}, x_{22}), (y_{32}, x_{32})\}$. Die erste Subsumption ist
hiernach gültig.

\begin{center}
\begin{minipage}{0.4\textwidth}
\begin{tikzpicture}[->,>=stealth',shorten >=5pt, shorten <=5pt,auto,node distance=1.5cm,
  main node/.style={circle,inner sep=2pt,minimum
  size=1pt,fill,=black,draw,font=\sffamily\Large\bfseries},
  not main node/.style={circle,inner sep=2pt,minimum size=1pt,fill=black,draw,font=\sffamily\Large\bfseries}]

  \node[main node,label=west:$x_1$,label=east:{$A$}] (1) {};
  \node[main node,label=west:$x_{21}$,label=east:{$B$}] (2) [below left of=1] {};
  \node[main node,label=west:$x_{31}$,label=east:{$C$}] (3) [below of=2] {};
  \node[main node,label=west:$x_{22}$,label=east:{$B$}] (4) [below right of=1] {};
  \node[not main node,label=west:$x_{32}$,label=east:{$D$}] (5) [below of=4] {};

  \path[every node/.style={font=\sffamily\small}]
    (1) edge node [right] {} (2)
		edge node [left]  {} (4)
	(2) edge node [right] {} (3)
	(4) edge node [right] {} (5);
\end{tikzpicture}
\end{minipage}
\begin{minipage}{0.4\textwidth}
\begin{tikzpicture}[->,>=stealth',shorten >=5pt, shorten <=5pt,auto,node distance=1.5cm,
  main node/.style={circle,inner sep=2pt,minimum
  size=1pt,fill,=black,draw,font=\sffamily\Large\bfseries},
  not main node/.style={circle,inner sep=2pt,minimum size=1pt,fill=black,draw,font=\sffamily\Large\bfseries}]

  \node[main node,label=west:$y_1$,label=east:{$A$}] (1) {};
  \node[main node,label=west:$y_{21}$,label=east:{$B$}] (2) [below of=1] {};
  \node[main node,label=west:$y_{22}$,label=east:{$C,D$}] (4) [below of=2] {};

  \path[every node/.style={font=\sffamily\small}]
    (1) edge node [right] {} (2)
	(2) edge node [right] {} (4);
\end{tikzpicture}
\end{minipage}
\end{center}

Es gibt lediglich zwei Relationen, die für eine Simulation infrage kommen, da
zwangsläufig einer der beiden Zweige des linken Modells gewählt werden muss.
Die beiden möglichen Relationen $R_1 = \{(y_1, x_1), (y_{21}, x_{21}), (y_{22},
x_{31})\}$ und $R_2 = \{ (y_1,  x_1), (y_{21}, x_{22}), (y_{22}, x_{32})\}$
sind jedoch keine Simulationen, da hierfür sowohl an Punkt $x_{31}$ als auch an
Punkt $x_{32}$ $C$ \emph{und} $D$ gelten müssten. Folglich ist die zweite
Subsumption nicht gültig.
