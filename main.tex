\documentclass[a4paper]{scrartcl}
\usepackage{fancyhdr}
\pagestyle{plain}
\usepackage{fullpage}
\usepackage[ngerman]{babel}
\usepackage{tikz}
\usepackage{enumerate}
\usepackage{listings}
\usepackage[colorlinks=true,linkcolor=black]{hyperref}
\usetikzlibrary{calc}
\usetikzlibrary{shadows}
\usetikzlibrary{matrix}
\usetikzlibrary{automata}
\usetikzlibrary{fadings}
\usetikzlibrary{decorations}
\usetikzlibrary{decorations.pathmorphing}
\usetikzlibrary{decorations.pathreplacing}
\usetikzlibrary{positioning}
\usetikzlibrary{arrows}
\usepackage{dsfont}
\usepackage{amsmath, amsthm, amssymb}
\usepackage{algorithm}
\usepackage{algpseudocode}
\usepackage{color}
%\usepackage{scrpage2}
\usepackage[T1]{fontenc}
\usepackage[utf8]{inputenc}
%\usepackage[math,light]{anttor}
\usepackage{textcomp}
\usepackage{etoolbox}
\usepackage{bussproofs}
\usepackage{mathtools}
\usepackage{colonequals}

% Pretty captions!
%\usepackage{caption}
%\DeclareCaptionFont{white}{\color{white}}
%\DeclareCaptionFormat{listing}{\colorbox{gray}{\parbox{0.98\linewidth}{#1#2#3}}}
%\captionsetup[lstlisting]{format=listing,labelfont=white,textfont=white}

% Colours
\definecolor{Golden}{RGB}{255,208,115}

% Pretty page headings!
\pagestyle{fancy}
\addtolength{\voffset}{-1.8cm}
\fancyhf{}% sets both header and footer to nothing
\renewcommand{\headrulewidth}{0pt}
\setlength{\headheight}{1.5cm}
\setlength{\textheight}{23cm}
\setlength{\headsep}{1.5cm}

\newcounter{blatt}
\setcounter{blatt}{1}
\lhead{
Ulrich Dorsch\\
Christoph Rauch\\
Manuel Zerpies
}
\rhead{
$\mathcal{O}$ntologien im $\mathcal{S}$emantic $\mathcal{W}$eb, WS12/13 \\
Übungsblatt \theblatt
}
\chead{
\begin{tikzpicture}[remember picture,overlay]
 \coordinate (a) at ($(current page.north west) - (-2cm,2.5cm)$);
 \coordinate (b) at ($(a) + (4cm,0)$);
 \coordinate (c) at ($(current page.north east) - (2cm,2.5cm)$);
 \coordinate (d) at ($(c) - (3.2cm,0)$);
 \draw[black]
 decorate [decoration={random steps,segment length=3pt,amplitude=1pt}] %
 {(a) -- (b)}%
 decorate [decoration={random steps,segment length=5pt,amplitude=4pt}] %
 {-- (b) -- (d)}%
 decorate [decoration={random steps,segment length=3pt,amplitude=1pt}] %
 {-- (d) -- (c)};%
\end{tikzpicture}
}
\rfoot{\pagemark}

\makeatletter
% Different format for headings
\def\section{\@startsection{section}{1}%
  \z@{.7\baselineskip\@plus\baselineskip}{.5\baselineskip}%
  {\normalfont\Large\scshape\centering}}
% This does spacing around caption.
\setlength{\abovecaptionskip}{0.5em}
\setlength{\belowcaptionskip}{0.5em}
% This does justification (left) of caption.
\long\def\@makecaption#1#2{%
  \vskip\abovecaptionskip
  \sbox\@tempboxa{#2}%
  \ifdim \wd\@tempboxa >\hsize
    #2\par
  \else
    \global \@minipagefalse
    \hb@xt@\hsize{\box\@tempboxa\hfill}%
  \fi
  \vskip\belowcaptionskip}
\providerobustcmd*{\bigcupdot}{%
  \mathop{%
    \mathpalette\bigop@dot\bigcup
  }%
}
\newrobustcmd*{\bigop@dot}[2]{%
  \setbox0=\hbox{$\m@th#1#2$}%
  \vbox{%
    \lineskiplimit=\maxdimen
    \lineskip=-0.7\dimexpr\ht0+\dp0\relax
    \ialign{%
      \hfil##\hfil\cr
      $\m@th\cdot$\cr
      \box0\cr
    }%
  }%
}
\makeatother

\setkomafont{title}{\normalfont}
\setkomafont{pageheadfoot}{\normalfont}
\setkomafont{pagenumber}{\normalfont\scshape}
\setkomafont{disposition}{\normalfont\bfseries}

\widowpenalty=300
\clubpenalty=300

\floatname{algorithm}{Algorithmus}
\newcommand{\algorithmautorefname}{Algorithmus}
\renewcommand{\algorithmicrequire}{\textbf{Input:}}
\renewcommand{\algorithmicensure}{\textbf{Output:}}

\newcommand{\cm}[1]{\textcolor{magenta}{#1}}

\newcommand{\aufgabe}[2]{
    \newpage
    \section*{Aufgabe #1 -- #2}
    \addcontentsline{toc}{subsection}{Blatt \theblatt, Aufgabe #1 -- #2}}

\newcommand{\angabe}[1]{\begin{center}
	\colorbox{Golden}{\begin{minipage}%
	{0.95\textwidth}#1\end{minipage}}%
	\end{center}\vspace{0.5cm}}

\newcommand{\defeq}{\vcentcolon=}
\newcommand{\defin}[2]{#1 $\sqsubseteq$ #2}
\newcommand{\ando}{$\sqcap$\hspace{1pt}}
\newcommand{\kripke}[1]{\ensuremath{\mathfrak{#1}}}
\newcommand{\relation}[1]{\ensuremath{\mathcal{#1}}}
\newcommand{\dia}{\Diamond}
\newcommand{\anm}[1]{\RightLabel{\scriptsize{#1}}}
\newcommand{\el}{\ensuremath{\mathcal{EL}}}

\author{Ulrich Dorsch, Christoph Rauch, Manuel Zerpies}

\title{Ontologien im Semantic Web\\
       Wintersemester 2012/13 \\
       \small Übungsaufgaben}

\begin{document}
\maketitle
\tableofcontents
\aufgabe{1}{Tableaux-Algorithmus}
\angabe{%
Wenden Sie den Tableaux-Algorithmus auf die Formel
\[
  \left( (A \wedge B) \to \neg C \right) \wedge (\neg A \to C) \wedge (\neg B
  \to C) \wedge \neg (B \to \neg C)
\]
an. Ist die Formel erfüllbar? [Hinweis: Sie müssen zunächst alle Konnektive
durch $\neg$ und $\wedge$ dekodieren.]
}

\vspace{-2em}
\begin{align*}
  \varphi & = \left( (A \wedge B) \to \neg C \right) \wedge (\neg A \to C) \wedge (\neg B \to C) \wedge
        \neg (B \to \neg C)
\intertext{… Auflösen der Implikationen …}
    & = \left( \neg (A \wedge B) \vee \neg C \right) \wedge (\neg \neg A \vee C) \wedge (\neg \neg B \vee C)
        \wedge \neg (\neg B \vee \neg C) \\ 
\intertext{… und der Disjunktionen sowie der doppelten Negationen …}
    & = \neg \left( A \wedge B \wedge C \right) \wedge \neg (\neg A \wedge \neg C) \wedge \neg (\neg B \wedge \neg C)
        \wedge (B \wedge C)
\end{align*}
Aus der ersten und letzten Klammer ergibt sich in dieser Form offensichtlich, dass für eine erfüllende Belegung $A =
\bot, B = \top, C = \top$ gelten muss. Nach Überprüfung der restlichen Terme erhält man kurzerhand das Ergebnis, dass
\[
  \kappa : \mathcal{A} \to \mathbf{2} : A \mapsto \bot, \ B \mapsto \top, \ C \mapsto \top
\]
die Formel erfüllt. Der Tableaux-Algorithmus sollte uns im Folgenden also dasselbe berichten:

\begin{gather*}
    \big\{ \neg \left( A \wedge B \wedge C \right) \cm{\wedge} \neg (\neg A \wedge \neg C) \cm{\wedge} \neg (\neg B
          \wedge \neg C) \cm{\wedge} B \cm{\wedge} C \big\} \\
\intertext{… Regel $(\wedge)$ für alle $\wedge$-Konnektive auf oberster Ebene …}
    \big\{ \cm{\neg} \left( \cm{A} \wedge (B \wedge C) \right), \neg (\neg A \wedge \neg C), \neg (\neg B \wedge \neg
          C), B, C \big\} \\
\intertext{… Regel $(\neg\wedge)$ auf erstes Element (linke Seite gewählt) …}
    \big\{ \neg A, \cm{\neg} (\neg A \wedge \cm{\neg C}), \neg (\neg B \wedge \neg C), B, C \big\} \\
\intertext{… Regel $(\neg\wedge)$ auf zweites Element (rechte Seite gewählt) …}
    \big\{ \neg A, \cm{\neg \neg C}, \neg (\neg B \wedge \neg C), B, C \big\} \\
\intertext{… Regel $(\neg\neg)$ auf zweites Element …}
    \big\{ \neg A, C, \cm{\neg} (\cm{\neg B} \wedge \neg C), B, C \big\} \\
\intertext{… Regel $(\neg\wedge)$ auf drittes Element (linke Seite gewählt) …}
    \big\{ \neg A, C, \cm{\neg \neg B}, B \big\} \\
\intertext{… Regel $(\neg\neg)$ auf drittes Element …}
    \big\{ \neg A, C, B \big\}
\end{gather*}
Hier ist keine Regel mehr anwendbar, die Formel wird wie erwartet als erfüllbar erkannt mit oben beschriebener
Wahrheitsbelegung.
\aufgabe{2}{Resolution}
\angabe{%
Wenden Sie den Resolutionsalgorithmus auf die CNF
\[
  \big\{ \{\neg D, A, B\}, \{C, B, \neg A\}, \{\neg D, B\}, \{\neg B, A\}, \{D,
  C\}, \{\neg C, B, \neg A\}, \{D, B, \neg C\}, \{\neg B, \neg A\} \big\}
\]
an -- ist die CNF erfüllbar? Wenden Sie alternativ DPLL an -- ergeben sich Vereinfachungen?
}
\vspace{-2em}
\[
 \varphi = \big\{ \{\neg D, A, B\}, \{C, B, \neg A\}, \{\neg D, B\}, \{\neg B, A\}, \{D, C\}, \{\neg C, B, \neg A\},
            \{D, B, \neg C\}, \{\neg B, \neg A\} \big\}
\]
Aus der vierten und der letzten Klausel lässt sich sofort
\begin{prooftree}
    \AxiomC{$\{\neg B, \cm{A}\}$}
    \AxiomC{$\{\neg B, \cm{\neg A}\}$}
    \BinaryInfC{$\{\neg B\}$}
\end{prooftree}
resolvieren. Ebenso erhält man durch die dritte und fünfte Klausel
\begin{prooftree}
        \AxiomC{$\{\cm{\neg D}, B\}$}
        \AxiomC{$\{\cm{D}, C\}$}
    \BinaryInfC{$\{B, C\}$}
\end{prooftree}
Desweiteren kann man durch eine Folge von Anwendungen der Resolutionsregel folgende Klauseln erzeugen:
\begin{prooftree}
        \AxiomC{$\{\cm{\neg D}, A, B\}$}
        \AxiomC{$\{\cm{D}, B, \neg C\}$}
    \RightLabel{\scriptsize{1 und 7}}
    \BinaryInfC{$\{A, B, \neg C\}$}
\end{prooftree}
sowie
\begin{prooftree}
    \AxiomC{$\{\neg D, \cm{A}, B\}$}
    \AxiomC{$\{\neg C, B, \cm{\neg A}\}$}
    \RightLabel{\scriptsize{1 und 6}}
    \BinaryInfC{$\{\neg D, B, \neg C\}$}
    \AxiomC{$\{D, \cm{B}, \neg C\}$}
    \AxiomC{$\{\cm{\neg B}, \neg A\}$}
    \RightLabel{\scriptsize{7 und 8}}
    \BinaryInfC{$\{D, \neg C, \neg A\}$}
\BinaryInfC{$\{\neg A, B, \neg C\}$}
\end{prooftree}
Resolviert man diese nun in umgekehrter Reihenfolge, so ergibt sich die leere Klausel. Die Formel ist folglich
unerfüllbar.

\vspace{2em}

Nun soll noch der Algorithmus DPLL auf dieselbe Formel angewendet werden. Dazu wählen wir zuerst die Variable $B$. Im
Falle $B = \top$ wird $\varphi / B$ untersucht:
\[
  \varphi / B = \big\{ \{\cm{A}\}, \{D, C\}, \{\cm{\neg A}\} \big\}
\]
Anwenden der Optimierungsmethode \emph{Unit Propagation} ($\{A\} \in \varphi \Rightarrow \varphi := \varphi / A$)
liefert sofort die leere Klausel. Es muss also mit \emph{Backtracking} der zweite Fall, $B = \bot$, untersucht werden:
\[
  \varphi := \varphi / \neg B = \big\{ \{\neg D, A\}, \{C, \neg A\}, \{\cm{\neg D}\}, \{D, C\}, \{\neg C, \neg A\},
  \{D, \neg C\}, \{\neg A\} \big\}
\]
Der nächste Schritt im Algorithmus ist erneut die \emph{Unit Propagation}, dieses Mal mit dem Literal $\neg D$
\[
  \varphi := \varphi / \neg D = \big\{ \{C, \neg A\}, \{\cm{C}\}, \{\neg C, \neg A\}, \{\cm{\neg C}\}, \{\neg A\} \big\}
\]
Nach einer weiteren \emph{Unit Propagation} ergibt sich auch hier die leere Klausel. Der Algorithmus bricht ab, die
Formel ist nicht erfüllbar.

\aufgabe{3}{Maximale konsistente CNF}
\angabe{%
Man erinnere sich, dass eine Klauselmenge (also eine CNF) $\phi$
\emph{resolutionsabgeschlossen} heißt, wenn aus $C \cup \{A\}, D \cup \{\neg
A\} \in \phi$ stets $C \cup D \in \phi$ folgt. Wir nennen eine CNF $\phi$ \emph{konsistent}, wenn sich aus ihr durch wiederholtes Anwenden der Resolutionsregel \emph{nicht} die leere Klausel herleiten lässt, wenn sie also vom Resolutionsalgorithmus als erfüllbar erkannt wird. Gegeben eine endliche Menge $\mathcal{A}$ von Aussagenvariablen heißt $\phi$ eine \emph{CNF über $\mathcal{A}$}, wenn $\phi$ nur Aussagenvariablen aus $\mathcal{A}$ erwähnt, und \emph{maximal konsistent über $\mathcal{A}$}, wenn $\phi$ maximal unter den konsistenten CNF über $\mathcal{A}$ ist. Beweisen Sie ohne Verwendung des Vollständigkeitsbeweises aus der Vorlesung folgende Aussagen:
\begin{enumerate}
	\item Jede konsistente CNF über $\mathcal{A}$ ist enthalten in einer maximal konsistenten CNF über $\mathcal{A}$.
	\item Sei $\phi$ eine maximal konsistente CNF über $\mathcal{A}$.
		\begin{enumerate}
			\item Für $A \in \mathcal{A}$ gilt $\{A\} \not\in \phi$ genau dann, wenn $\{\neg A\} \in \phi$.
			\item $\phi$ ist resolutionsabgeschlossen.
			\item $\phi$ ist abgeschlossen unter Abschwächung von Klauseln, d.h. aus $C \in \phi$ und $C \subseteq D$ folgt $D \in \phi$.
			\item Für Klauseln $C, D$ gilt $C \cup D \in \phi$ genau dann, wenn $C \in \phi$ oder $D \in \phi$.
			\item Die durch
				\[ \kappa(A) = \top \Leftrightarrow \{A\} \in \phi \]
				  gegebene Wahrheitsbelegung $\kappa\ :\ \mathcal{A} \to \mathbf{2}$ erfüllt $\phi$.
		\end{enumerate}
	\item Der Resolutionsalgorithmus ist vollständig, d.h.\ jede konsistente CNF ist erfüllbar.
\end{enumerate}
}
\begin{enumerate}
    \item Da Maximalität von konsistenten CNF über die Teilmengenrelation definiert ist, also 
          \[
            \phi \text{ maximal konsistent } \Leftrightarrow \forall \psi\ .\ \phi \subseteq \psi \Rightarrow \psi = \phi,
          \]
          ist jede konsistente CNF Teilmenge einer maximal konsistenten CNF: Die Menge der konsistenten CNF ist
          endlich und da auf den konsistenten CNF mittels der Teilmengenrelation eine partielle Ordnung definiert
          wird, existiert für alle Teilmengen von miteinander vergleichbaren konsistenten CNF (im Sinne der
          Teilmengenrelation) ein Maximum.
    \item
        \begin{enumerate}
                \item Hier gilt es zum einen, zu zeigen, dass \emph{jedes} Literal $A$ aus $\mathcal{A}$ (entweder
                positiv oder negativ) als Klausel $\{A\}$ oder $\{\neg A\}$ in $\phi$ vorkommt und zum anderen, dass
                nicht beide Klauseln zugleich vorkommen können:

                Annahme: $\{A\}, \{\neg A\} \not\in \phi$ und $\phi$ maximal. Wird nun o.B.d.A.\ $\{A\}$ zur
                Klauselmenge $\phi$ hinzugefügt, so ändert dies nichts an der Konsistenz der CNF, da für Klauseln C
                folgt:
                \begin{itemize}
                        \item[]$A \in C$: Resolution mit $\{A\}$ nicht möglich
                        \item[]$\neg A \in C$: Resolution mit $\{A\}$ liefert $D := C \setminus \{\neg A\}$.
                                Angenommen, $D$ ließe sich nun zur leeren Klausel resolvieren, dann ließe sich $C$ zu
                                $\{\neg A\}$ resolvieren, was im Widerspruch zur Annahme steht. Daher muss $\phi$ unter
                                Hinzunahme von $D$ und $\{A\}$ konsistent bleiben.
                        \item[]$A, \neg A \not \in C$: Resolution mit $\{A\}$ nicht möglich.
                \end{itemize}
                Da in jedem dieser Fälle $\{A\}$ hinzugefügt werden konnte, kann $\phi$ nicht maximal sein. Es muss
                also $\{A\}$ oder $\{\neg A\}$ in $\phi$ enthalten sein.

                Annahme: $\{A\} \in \phi, \{\neg A\} \in \phi$: dann lassen sich diese beiden Klauseln zur leeren
                Klausel resolvieren und $\phi$ ist nicht konsistent.

                Damit gilt die Behauptung.

                \item Sei $\phi$ maximal konsistent und nicht resolutionsabgeschlossen. Dann kann durch Resolution eine
                weitere Klausel hinzugefügt werden, was der Annahme widerspricht, dass $\phi$ maximal ist.

                \item Erweitert man eine Klausel $C \in \phi$ um eine beliebige Menge von Literalen $X$, so gilt: da $C$
                sich nicht auf die leere Klausel resolvieren lässt (Konsistenz), kann $C \cup X$ nicht auf $X$
                resolviert werden, da immer ein Teil von $C$ bei der Resolution übrig bleibt. Damit kann $C \cup X$
                auch nicht zur leeren Klausel resolviert werden und auch keine andere Klausel durch Resolution mit $C
                \cup X$ und $\phi \cup \{C \cup X\}$ ist konsistent.

                \item Die Richtung \glqq$\Leftarrow$\grqq\ sofort aus c), da sowohl $C$ als auch $D$ Teilmengen von $C
                \cup D$ sind.

                \glqq$\Rightarrow$\grqq: angenommen $C \cup D \in \phi, C \not\in \phi, D \not\in \phi,\ \phi$ maximal
                konsistent. Dann ist $\phi\,\cup \{C\}$ nicht konsistent, d.h. $C$ führt durch Resolution zur leeren
                Menge. Dies bedeutet jedoch für $\phi$, dass $C \cup D$ dort zu $(C \cup D) \setminus C$ (also einer
                Teilmenge von $D$) resolviert werden kann. Nach c) ist damit $D$ in $\phi$, was im Widerspruch zur
                Annahme steht.

                \item Nach a) gilt für jedes $A \in \mathcal{A}$ entweder $\{A\} \in \phi$ oder $\{\neg A\} \in \phi$.
                Die gegebene Belegung erfüllt all diese einelementigen Klauseln. Nach d) kann jede Klausel aus $\phi$
                zerlegt werden in $C = D \cup E$, wobei mindestens eine der beiden Teilklauseln in $\phi$ ist. Durch
                wiederholte Anwendung kann man so jede Klausel $C$ in eine Vereinigung von einelementigen Klauseln
                unterteilen, von denen mindestens eine in $\phi$ und damit durch $\kappa$ erfüllbar ist und damit auch
                $C$.
        \end{enumerate}
    \item Nach 2. e) existiert für jede maximale konsistente CNF eine erfüllende Belegung und nach 1. ist jede
    konsistente CNF $\psi$ eine Teilmenge einer geeigneten maximalen konsistenten CNF $\phi$. Damit gilt $\kappa_{\psi}
    = \kappa_{\phi}$, da $\phi$ alle Klauseln von $\psi$ enthält.
\end{enumerate}
\vfill
\begin{figure}[!ht]
\centering
\includegraphics[height=16cm]{ontorab.jpg}\\
{\small Ontology of a Rabbit \textcopyright 2012 Kit Lang}
\end{figure}
\vfill

\stepcounter{blatt}
\aufgabe{1}{Atomare Axiomenregel}
\newcommand{\Tk}{T_\text{komplex}}
\newcommand{\Ta}{T_\text{atomar}}
Bezeichne $\Tk$ den ursprünglichen Kalkül mit komplexer Axiomenregel $(\phi, \neg \phi, \Gamma / \bot)$ und $\Ta$ den
neuen Kalkül mit atomarer Axiomenregel und sei $\#\phi$ die Anzahl der Konnektive der Formel $\phi$.

N.B.: Außer der Axiomenregel gibt es keine Regel, die \glqq kommaübergreifend\grqq\ arbeitet; dies bedeutet, dass
Regeln, die lediglich auf $\Gamma$ angewendet werden, ignoriert werden können.

Zu zeigen ist, dass $\Ta$ genau dann ein Label $\Gamma$ ablehnt, wenn $\Gamma$ von $\Tk$ abgelehnt wird.

\begin{itemize}
        \item[\glqq$\Rightarrow$\grqq] Lässt sich die atomare Axiomenregel anwenden, so auch die komplexe Axiomenregel.
        \item[\glqq$\Leftarrow$\grqq] Induktion über die Struktur von $\phi$:
                        \paragraph{I.A.} $\#\phi = 0 \Rightarrow \phi = A, A$ Literal $\Rightarrow$ $\Tk$ und $\Ta$ lehnen den
                        Label $\phi, \neg \phi, \Gamma$ ab
                        \paragraph{I.V.} $\forall \phi, \ \#\phi < n$ gelte bereits:
                                \[ \Tk \text{ lehnt } \phi, \neg \phi, \Gamma \text{ ab } \Rightarrow \Ta \text{ lehnt }
                                \phi, \neg \phi, \Gamma \text{ ab } \]
                        \paragraph{I.S.} Seien $\phi_1, \phi_2$ zwei Formeln mit $\#\phi_1 = k_1 < n, \#\phi_2 = k_2 < n$.
                        \begin{itemize}
                                \item[$i)$] $\phi = \phi_1 \wedge \phi_2$ und $\Tk$ lehnt den Label $\phi, \neg \phi, \Gamma$
                                        ab. Folgende Zweige können durch Regelanwendungen im Kalkül $\Ta$ entstehen:

                                        \begin{minipage}{0.4\textwidth}
                                        \begin{prooftree}
                                                \AxiomC{$\phi_1 \wedge \phi_2, \neg (\phi_1 \wedge \phi_2), \Gamma$}
                                                \UnaryInfC{$\phi_1, \phi_2, \neg (\phi_1 \wedge \phi_2), \Gamma$}
                                                \UnaryInfC{$\phi_1, \phi_2, \neg \phi_1, \Gamma$}
                                                \RightLabel{\scriptsize{nach I.V.}}
                                                \UnaryInfC{$\bot$}
                                        \end{prooftree}\end{minipage}\begin{minipage}{0.4\textwidth}
                                        \begin{prooftree}
                                                \AxiomC{$\phi_1 \wedge \phi_2, \neg (\phi_1 \wedge \phi_2), \Gamma$}
                                                \UnaryInfC{$\phi_1, \phi_2, \neg (\phi_1 \wedge \phi_2), \Gamma$}
                                                \UnaryInfC{$\phi_1, \phi_2, \neg \phi_2, \Gamma$}
                                                \RightLabel{\scriptsize{nach I.V.}}
                                                \UnaryInfC{$\bot$}
                                        \end{prooftree}
                                        \end{minipage}

                                    In beiden Fällen wird der Label auch von $\Ta$ abgelehnt.

                                \item[$ii)$] $\phi = \neg\phi_1$ und $\Tk$ lehnt den Label $\neg\phi_1, \neg\neg\phi_1, \Gamma$
                                        ab. Im Kalkül $\Ta$ erhält man:
                                        \begin{prooftree}
                                                \AxiomC{$\neg\phi_1, \neg\neg\phi_1, \Gamma$}
                                                \RightLabel{\scriptsize{$(\neg\neg)$}}
                                                \UnaryInfC{$\neg\phi_1, \phi_1, \Gamma$}
                                                \UnaryInfC{$\bot$}
                                        \end{prooftree}

                                    Der Label wird also auch von $\Ta$ abgelehnt.
                        \end{itemize}
                        Da es nur die Konnektive $\wedge$ und $\neg$ gibt, folgt die Behauptung.
\end{itemize}

\newpage
\aufgabe{2}
Da der Label $\Gamma$ endlich viele Konnektive hat und jede Regel die Anzahl der Konnektive verringert, führen alle
Kombinationen von Regelanwendungen in endlicher Zeit zur Terminierung. Damit ist der Begriff eines \emph{erfolgreichen}
Labels wohldefiniert.

Der Begriff \emph{erfolgreich} wird also verallgemeinert, sodass ein Baum von Regelanwendungen entsteht, bei dem die
Pfade von der Wurzel zu jedem Blatt Folgen von Regelanwendungen repräsentieren, deren Anwendungen auf $\Gamma$ zu einer
erfolgreichen Terminierung des Tableaualgorithmus führen.

Zu zeigen: Label erfolgreich $ \ \Leftrightarrow \ \exists $ erfolgreicher Zweig
Beweis:
\begin{itemize}
        \item [$\Rightarrow$] Nach Definition.
        \item [$\Leftarrow$] Label $\Gamma$ hat im Tableaualgorithmus einen erfolgreichen Zweig.
                
                Induktion über die Baumhöhe:
                \paragraph{I.A.} Höhe 0: $\Gamma = A_1, \dots, A_m$, $A_i$ verschiedene Literale (negativ oder positiv),
                d.h. es ist keine weitere Regel anwendbar $\Rightarrow \ \Gamma$ erfolgreich (\emph{vacuously}).
                \paragraph{I.V.} Für alle Label $\Gamma$ der Höhe $k < n$ mit erfolgreichem Zweig gelte die Behauptung.
                \paragraph{I.S.} Höhe $n$.
                \begin{itemize}
                        \item[$(\wedge)$] $\Gamma = \varphi \wedge \psi, \Xi \ \vdash \ \phi, \psi, \Xi \ \Rightarrow $ Höhe
                                $< n$

                                Nach I.V. gilt: $\varphi, \psi, \Xi$ ist erfolgreicher Label. Somit führt auch
                                die Regelanwendung bei $\Gamma$ zu einer erfolgreichen Konklusion.
                        \item[$(\neg\neg)$] $\Gamma = \neg \neg \varphi, \Xi \ \vdash \ \varphi, \Xi \ \Rightarrow $ Höhe
                                $< n$

                                Nach I.V. gilt: $\varphi, \Xi$ ist erfolgreicher Label. Somit führt auch
                                die Regelanwendung bei $\Gamma$ zu einer erfolgreichen Konklusion.
                        \item[$(\neg\wedge)$] $\Gamma = \neg (\varphi \wedge \psi), \Xi \ \vdash \ \neg \varphi, \Xi \|
                                \neg \psi, \Xi \ \Rightarrow $ Höhe jeweils
                                $< n$

                                O.B.d.A. ist $\neg \varphi, \Xi$ erfolgreicher Zweig von $\Gamma$. Nach I.V. gilt:
                                $\neg \varphi, \Xi$ ist erfolgreicher Label. Somit führt auch diese Regelanwendung bei
                                $\Gamma$ zu mindestens einer erfolgreichen Konklusion.
                \end{itemize}
                Somit hat $\Gamma$ für jede mögliche Regelanwendung eine erfolgreiche Konklusion.
\end{itemize}
Damit ist $\Gamma$ ein erfolgreicher Label.

\newpage
\aufgabe{3}{Reduktion von BDDs}
Die Formel $(x \vee y) \wedge (z \vee w)$ kann dargestellt werden als baumförmiges BDD:

\begin{center}
\begin{tikzpicture}[
        nterm/.style={circle,draw,anchor=north},
        term/.style={rectangle,draw,anchor=north},
        level distance=0.76cm, growth parent anchor=south, sibling distance=3cm
        ]
        \node (x) [nterm] {$x$} [->]
            child{ [sibling distance=1cm] node (y) [nterm] {$y$}
            child{ node (a0) [term] {$0$} edge from parent node[left,scale=.75] {$0$} }
                        child{ node (z1) [nterm,fill=green!50] {$z$}
                                    child{ node (w1) [nterm,fill=green!50] {$w$}
                                                child{ node (b0) [term,fill=green!50] {$0$} edge from parent node[left,scale=.75] {$0$} }
                                                child{ node (a1) [term,fill=green!50] {$1$} edge from parent node[right,scale=.75] {$1$} }
                                    edge from parent node[left,scale=.75] {$0$} }
                                    child { node (b1) [term,fill=green!50] {$1$} edge from parent node[right,scale=.75] {$1$} }
                        edge from parent node[right,scale=.75] {$1$} }
            edge from parent node[above left,scale=.75] {$0$} }
            child{ [sibling distance=1cm] node (z2) [nterm,fill=green!50] {$z$}
                        child{ node (w2) [nterm,fill=green!50] {$w$}
                                    child{ node[fill=green!50] (c0) [term] {$0$} edge from parent node[left,scale=.75] {$0$} }
                                    child{ node[fill=green!50] (c1) [term] {$1$} edge from parent node[right,scale=.75] {$1$} }
                               edge from parent node[left,scale=.75] {$0$} }
                        child{ node (d1) [term,fill=green!50] {$1$} edge from parent node[right,scale=.75] {$1$} }
        edge from parent node[above right,scale=.75] {$1$} };
\end{tikzpicture}
\end{center}

Die farbig markierten Untergraphen der beiden $z$-Knoten sind isomorph. Somit kann einer der $z$-Knoten entfernt werden,
wobei die eingehende Kante auf den anderen $z$-Knoten umgebogen wird. Dadurch entsteht der links abgebildete Graph:

\begin{minipage}{0.4\textwidth}
\begin{center}
\begin{tikzpicture}[
        p/.style={scale=.75,->,draw},
        nterm/.style={circle,draw,anchor=north},
        term/.style={rectangle,draw,anchor=north},
        level distance=0.76cm, growth parent anchor=south, sibling distance=2cm
        ]
        \node (x) [nterm] {$x$};
        \node [nterm,below left=of x] (y) {$y$};
        \node [nterm,below right=of x] (z) {$z$};
        \node [term,below left=of y,fill=red!50] (a0) {$0$};
        \node [nterm,below left=of z] (w) {$w$};
        \node [term,below right=of z,fill=blue!50] (a1) {$1$};
        \node [term,below left=of w,fill=red!50] (b0) {$0$};
        \node [term,below right=of w,fill=blue!50] (b1) {$1$};
        \path [p] (x) -- node[above left] {$0$} (y);
        \path [p] (x) -- node[above right] {$1$} (z);
        \path [p] (y) -- node[above left] {$0$} (a0);
        \path [p] (y) -- node[above] {$1$} (z);
        \path [p] (z) -- node[above left] {$0$} (w);
        \path [p] (z) -- node[above right] {$1$} (a1);
        \path [p] (w) -- node[above left] {$0$}  (b0);
        \path [p] (w) -- node[above right] {$1$}  (b1);
\end{tikzpicture}
\end{center}
\end{minipage}\hfill\begin{minipage}{0.4\textwidth}
\begin{center}
\begin{tikzpicture}[
        p/.style={scale=.75,->,draw},
        nterm/.style={circle,draw,anchor=north},
        term/.style={rectangle,draw,anchor=north},
        level distance=0.76cm, growth parent anchor=south, sibling distance=2cm
        ]
        \node (x) [nterm] {$x$};
        \node [nterm,below left=of x] (y) {$y$};
        \node [nterm,below right=of x] (z) {$z$};
        \node [nterm,below left=of z] (w) {$w$};
        \node [term,below left=of w,fill=red!50] (b0) {$0$};
        \node [term,below right=of w,fill=blue!50] (b1) {$1$};
        \path [p] (x) -- node[above left] {$0$} (y);
        \path [p] (x) -- node[above right] {$1$} (z);
        \path [p] (y) -- node[left] {$0$} (b0);
        \path [p] (y) -- node[above] {$1$} (z);
        \path [p] (z) -- node[above left] {$0$} (w);
        \path [p] (z) -- node[right] {$1$} (b1);
        \path [p] (w) -- node[above left] {$0$}  (b0);
        \path [p] (w) -- node[above right] {$1$}  (b1);
\end{tikzpicture}
\end{center}
\end{minipage}

Nach Entfernen der doppelten terminalen Knoten erhält man den rechten Graphen, das vollständig reduzierte BDD.

\aufgabe{4}{Erfüllbarkeit von BDDs}
Ein BDD ist erfüllbar, wenn es einen Pfad von der Wurzel $r$ zu einem Terminal mit Label $1$ gibt. Die Existenz eines
terminalen Knotens mit Label $1$ ist sogar hinreichend:

Gäbe es einen terminalen Knoten $1$ ohne, dass ein Pfad von der Wurzel zu ihm existiert, so gäbe es außer der Wurzel
mindestens noch einen initialen Knoten, was im Widerspruch zur Definition eines BDD steht.

$\Rightarrow$ Algorithmus: Durchsuchen der Knotenmenge nach terminalem Knoten mit Label $1$. Die Laufzeit ist linear in
der Länge der Eingabe, der Algorithmus arbeitet ohne Platzoverhead.

\stepcounter{blatt}
\include{uebung3}
\stepcounter{blatt}
% vim: set filetype=tex:
\aufgabe{1}{Modellierung terminologischen Wissens}

\newcommand{\defin}[2]{#1 $\sqsubseteq$ #2}
\newcommand{\ando}{$\sqcap$\hspace{1pt}}
\newcommand{\kripke}[1]{\ensuremath{\mathcal{#1}}}
\newcommand{\relation}[1]{\ensuremath{\mathcal{#1}}}

Atomare Konzepte:
\begin{itemize}
	\item Album
	\item Song
	\item Genere, und diverse Subgenres
	\item Single
	\item Compilation
\end{itemize}

SuccessfulSong $\sqsubseteq$ Song 	$\sqcap$ $\exists$\hspace{1pt}appearsOn.Album
									$\sqcap$ $\exists$\hspace{1pt}appearsOn.Single
									$\sqcap$
									$\exists$\hspace{1pt}appearsOn.Compilation\\

SuccessfulArtist $\sqsubseteq$ Artist $\sqcap$
$\exists$\hspace{1pt}hasWritten.successfulSong\\

Artist $\sqsubseteq$ $\exists$\hspace{1pt}hasWritten.Song\\

Was nicht geht:
\begin{itemize}
	\item Beschreiben, dass ein Artist Songs in versch. Genres geschrieben hat.
	\item Zählen (z. B. eine Definition eines erfolgreichen Artists durch die
		Anzahl der Veröffentlichungen)
	\item \defin{BestOf}{Compilation \ando
		$\forall$contains.$\forall$writtenBy.\{\textit{sameauthor}\}}
	\item Selbstreferenzen
\end{itemize}

\aufgabe{2}{Verwendung von Werkzeugen}

\aufgabe{3}{Schließen in Modallogik}
Zz: Wenn der Label oberhalb des Regelstrichs gültig ist, so ist auch der Label
unterhalb des Regelstrichs gültig.



Es gilt: $\forall x \in M.\kripke{M},x\vDash \neg \phi_1,\dots,\neg \phi_n,\phi_0$
Dies heißt, dass entweder \[\kripke{M},x \vDash \phi_0\] oder \[\kripke{M},x
\nvDash \phi_0 \wedge \exists 0 < i \leq n.\kripke{M},x_j \nvDash \phi_i\] gelten muss.

Wir wählen nun ein beliebiges, aber festes, $x$ und betrachten dessen
Nachfolger:
\begin{enumerate}
	\item Für $x$ existiert kein Nachfolger, dann gilt trivialerweise für $x$
		$\Box \phi_0$.
	\item $x$ besitzt $k$ Nachfolger ($x_1, \dots, x_k$)
		\begin{enumerate}
			\item Wenn für alle Nachfolger $x_j$ ($j \in \{1, \dots, k\}$)
				\kripke{M}, $x_j \vDash \phi_0$
				Dann gilt:
				\[\kripke{M}, x \vDash \Box \phi_0\]
			\item Wenn für mindestens einen Nachfolger $x_j$ ($j \in \{1,
				\dots,k\}$)\kripke{M},$x_j \nvDash \phi_0$ gilt $\exists 0 < i \leq n.\kripke{M},x_j \nvDash
				\phi_i$, da für $x_j$ ebenfalls der Label oberhalb des
				Regelstriches gültig ist. Damit existiert ein Nachfolger von
				$x$, der nicht $\phi_i$ für ein $i$ erfüllt. Somit gilt $\neg
				\Box \phi_i$.
		\end{enumerate}
\end{enumerate}
Da $x$ beliebig ist die Behauptung korrekt.

\aufgabe{4}{(Keine) Bisimilarität}
\begin{center}
\begin{minipage}{0.4\textwidth}
\begin{tikzpicture}[->,>=stealth',shorten >=5pt, shorten <=5pt,auto,node distance=1.5cm,
  main node/.style={circle,inner sep=2pt,minimum size=1pt,fill,=black,draw,font=\sffamily\Large\bfseries,label=east:P},
  not main node/.style={circle,inner sep=2pt,minimum size=1pt,fill=black,draw,font=\sffamily\Large\bfseries,label=east:¬P}]

  \node[main node,label=west:$x_1$] (1) {};
  \node[main node,label=west:$x_2$] (2) [below of=1] {};
  \node[main node,label=west:$x_{31}$] (3) [below left of=2] {};
  \node[not main node,label=west:$x_{32}$] (4) [below right of=2] {};

  \path[every node/.style={font=\sffamily\small}]
    (1) edge node [right] {} (2)

	(2) edge node [right] {} (3)
        edge node [right] {} (4);
\end{tikzpicture}
\end{minipage}
\begin{minipage}{0.4\textwidth}
\begin{tikzpicture}[->,>=stealth',shorten >=5pt, shorten <=5pt,auto,node distance=1.5cm,
  main node/.style={circle,inner sep=2pt,minimum size=1pt,fill,=black,draw,font=\sffamily\Large\bfseries,label=0:P},
  not main node/.style={circle,inner sep=2pt,minimum size=1pt,fill=black,draw,font=\sffamily\Large\bfseries,label=0:¬P}]

  \node[main node,label=west:$y_1$] (1) {};
  \node[main node,label=west:$y_{21}$] (2) [below left of=1] {};
  \node[main node,label=west:$y_{31}$] (3) [below of=2] {};
  \node[main node,label=west:$y_{22}$] (4) [below right of=1] {};
  \node[not main node,label=west:$y_{32}$] (5) [below of=4] {};

  \path[every node/.style={font=\sffamily\small}]
    (1) edge node [right] {} (2)
		edge node [left]  {} (4)
	(2) edge node [right] {} (3)
	(4) edge node [right] {} (5);
\end{tikzpicture}
\end{minipage}
\end{center}

Gegeben sind die beiden baumförmigen Kripkemodelle \kripke{M} (links) und
\kripke{N} (rechts).
Diese sind bisimilar, wenn eine Relation $S \subseteq M \times N$ existiert und
diese sowie ihre Umkehrrelation $S^-$ Simulationen sind.

Zunächst zeigen wir, dass eine solche Relation $S$ nur existieren kann, wenn
$x_1 S y_1$ gilt. Gilt nämlich oBdA. $x_1 S y_i, i \neq 1$ dann folgt durch
Betrachtung der Nachfolger, wie nach Definitionsteil (ii), dass auf der rechten
Seite nicht ausrechend Nachfolger vorhanden sind.

Beweisführung durch Überprüfung der Definition von Simulation:

Nicht Modal:

Es gilt: $\kripke{M}, x_1 \vDash P$ und $\kripke{N},y_1 \vDash P$ (i). Für alle
Nachfolger von $x_1$, $x_2$, muss nun gelten: $\exists w.y_1 R w \wedge x_2 S w$
(ii).
\begin{enumerate}
	\item $w = y_{21}$ erfüllt diese Bedingung, da in beiden Punkten $P$ gilt.
		Überprüfe nun erneut (ii) bei dem Nachfolger $x_{32}$ von $x_2$. Nach
		Definition muss wiederum gelten: $\exists w'.y_{21} R w' \wedge x_{32} S
		w'$. Im Modell \kripke{N} kann nur $w' = y_{31}$ gelten. Damit ist $S$
		keine Simulation, da $\kripke{M}, x_{32} \nvDash P$ und $\kripke{N},
		y_{32} \vDash P$
	\item Für $w = y_{22}$ ist der Beweis analog bei Wahl Nachfolger $x_{31}$
		von $x_2$.
\end{enumerate}

Im modalen Fall erfolgt der Beweis analog mithilfe des
Bisimulationsinvarianzsatzes, wenn $P$ eine modale Formel ist.


\stepcounter{blatt}
% vim: set filetype=tex:
\newcommand{\defin}[2]{#1 $\sqsubseteq$ #2}
\newcommand{\ando}{$\sqcap$\hspace{1pt}}
\newcommand{\kripke}[1]{\ensuremath{\mathcal{#1}}}
\newcommand{\relation}[1]{\ensuremath{\mathcal{#1}}}
\newcommand{\dia}{\lozenge}

\aufgabe{1}{Modellierung terminologischen Wissens}

\angabe{%
Verwenden Sie den Tableaukalkül, um folgende Formeln auf Erfüllbarkeit zu prüfen:
\begin{equation*}
\Box(\dia p) \wedge \Box(p \rightarrow{} (\dia q \vee \dia r)) \wedge
\dia p \wedge \dia \neg p \wedge \Box \Box ((q \vee r) \rightarrow{} \neg p)
\wedge \Box (\neg p \rightarrow{} \Box \neg p)
\label{erfbar1}
\end{equation*}
\begin{equation*}
	\Box(p \rightarrow{} (\dia q \vee \dia r)) \wedge (\dia p \vee \dia \neg p)
	\wedge \Box(\neg p \rightarrow{} \Box \neg p) \wedge \Box \Box (p \wedge
	\neg r)
	\label{erfbar2}
\end{equation*}
}

Durch Umformen der ersten gegebenen Formel erhalten wir die folgende vereinfachte
Formel:
\begin{center}
$ \Box(\dia p) \wedge \Box\neg(p \wedge (\neg\dia q \wedge \neg\dia r)) \wedge
\neg\Box p \wedge \neg\Box\neg p \wedge \Box \Box \neg(\neg(\neg q \wedge \neg r)
\wedge  p) \wedge \Box \neg(\neg p \wedge \neg\Box\neg  p) $
\end{center}

Ab jetzt lassen sich die bekannten Tableau-Regeln anwenden.

\newcommand{\anm}[1]{\RightLabel{\scriptsize{#1}}}

(Möglichkeit 1)
\begin{prooftree}
\AxiomC{$ \Box(\dia p) , \Box\neg(p \wedge (\neg\dia q \wedge \neg\dia r)) ,
\neg\Box p , \neg\Box\neg p , \Box \Box \neg(\neg(\neg q \wedge \neg r)
\wedge  p) , \Box \neg(\neg p \wedge \neg\Box\neg  p) $}
\anm{$(\neg\Box)$}

\UnaryInfC{$ \dia p , \neg(p \wedge (\neg\dia q \wedge \neg\dia r)) ,
\neg\Box p , \neg\neg p , \neg(\neg(\neg q \wedge \neg r)
\wedge  p) , \neg(\neg p \wedge \neg\Box\neg  p) $}
\anm{$(\neg\Box)$}

\UnaryInfC{$ \dia p , \neg(p \wedge (\neg\dia q \wedge \neg\dia r)) ,
\neg p , \neg\neg p , \Box\neg(\neg(\neg q \wedge \neg r)
\wedge  p) , \neg(\neg p \wedge \neg\Box\neg  p) $}
\anm{$(\neg\neg)$}

\UnaryInfC{$ \dia p , \neg(p \wedge (\neg\dia q \wedge \neg\dia r)) ,
\neg p , p , \Box\neg(\neg(\neg q \wedge \neg r)
\wedge  p) , \neg(\neg p \wedge \neg\Box\neg  p) $}
\anm{clash}

\UnaryInfC{$\bot$}
\end{prooftree}

(Möglichkeit 2)
\begin{prooftree}
\AxiomC{$ \Box(\dia p) , \Box\neg(p \wedge (\neg\dia q \wedge \neg\dia r)) ,
\neg\Box p , \neg\Box\neg p , \Box \Box \neg(\neg(\neg q \wedge \neg r)
\wedge  p) , \Box \neg(\neg p \wedge \neg\Box\neg  p) $}
\anm{$(\neg\Box)$}

\UnaryInfC{$ \dia p , \neg(p \wedge (\neg\dia q \wedge \neg\dia r)) ,
\neg\Box p , \neg\neg p , \neg(\neg(\neg q \wedge \neg r)
\wedge  p) , \neg(\neg p \wedge \neg\Box\neg  p) $}
\end{prooftree}

\begin{prooftree}
\AxiomC{$ \dia p , \neg(p \wedge (\neg\dia q \wedge \neg\dia r)) ,
\neg\Box p , \neg\neg p , \neg(\neg(\neg q \wedge \neg r)
\wedge  p) ,  p$}
\end{prooftree}

\begin{prooftree}
\AxiomC{$\dia p , \neg(p \wedge (\neg\dia q \wedge \neg\dia r)) ,
\neg\Box p , \neg\neg p , \neg(\neg(\neg q \wedge \neg r)
\wedge  p) ,  \Box\neg  p $}
\end{prooftree}

Durch umformen der zweiten gegebenen Formel erhält man folgende einfache Formel:

\begin{center}
	$ \Box \neg(\neg p \wedge (\Box q \wedge \Box r)) \wedge \neg (\Box p \wedge
	\Box\neg p) \wedge \Box \neg (\neg p \wedge \Box\neg p) \wedge \Box\Box(p
	\wedge \neg r)$
\end{center}

Ab jetzt lassen sich die Tableau-Regeln anwenden.



\aufgabe{2}{Große Modelle}
\angabe{%
Gegeben seien Atome $q_1,q_2,\dots$. Man definiere die Familie von Formeln
$\phi_n$ durch
\begin{equation*}
\phi_n = \bigwedge^{n}_{i=0} \Box^i(\dia q_{i+1} \wedge \dia \neg
q_{i+1}) \wedge \displaystyle \bigwedge_{i=1}^{n} \Box^i(\Box^{\leq n - i}((q_i
\rightarrow \Box q_i) \wedge (\neg q_i \rightarrow{} \Box \neg q_i))).
\label{bigmodel}
\end{equation*}
Zeigen Sie, dass jedes Modell von $\phi_n$ mindestens $2^n$ Elemente hat. Geben
Sie eine polynomielle Abschätzung für die Größe von $\phi_n$ in Abhängigkeit
von $n$ an.
}

Sei $\kripke{M}$ ein Modell für $\phi_n$, $x\in M$ mit $\kripke{M}, x \vDash
\phi_n$.

Die erste Konjunktion in $\phi_n$ stellt sicher, dass jeder Punkt des
Modells, der von $x$ in maximal $n$ Schritten erreichbar ist, mindestens zwei
verschiedene Nachfolger besitzt.

Durch die zweite Konjunktion wird ein Atom $q_i$ (bzw. seine Negation $\neg q_i$)
von einem Punkt $y$ der von $x$ ausgehend in mindestens $i$ und maximal $n$
Schritten erreichbar ist auf alle seine Nachfolger propagiert; somit folgt,
dass von $y$ ab alle erreichbaren Punkte $q_i$ (bzw. $\neg q_i$) erfüllen,
solange sie nicht weiter als $n+1$ Schritte von $x$ entfernt sind.

Dies bedeutet, dass von $x$ aus im $i$-ten Schritt entweder $q_i$ oder $\neg
q_i$ gewählt werden kann und diese Wahl sich bis zum Schritt $n$ \glqq
gemerkt\grqq wird; damit folgt, dass im $n$-ten Schritt für die $n$ Atome
jeweils eine Wahl getroffen wurde, sodass diese Kombinationen jeweils paarweise
verschieden sind (und nicht gleichzeitig an einem Punkt in einem Kripkemodell
gelten können). Die Anzahl dieser Kombinationen ist $2^n$. Damit folgt, dass
jedes Modell von $\phi_n$ mindestens $2^n$ Elemente hat.

Abschätzung der Größe von $\phi_n$:
\[
\sum_{i=0}^{n}(i + 4) + n + 1 + \sum_{i=1}^{n}\left(i + \left(\sum_{j=0}^{n-i}
j + 7\right) + (n - i + 1) \right) + n = \frac{5+(65 n)}{6}+5 n^2+\frac{n^3}{6}
\in \mathcal{O}(n^3)
\]
\aufgabe{3}{Keine großen Modelle}
\angabe{$S5$ bezeichne die Modallogik der Kripkerahmen, deren Zustandsübergangsrelation transitiv, reflexiv
	und symmetrisch ist. Zeigen Sie, dass die Methode von Aufgabe~2 zur Widerlegung der polynomiellen
	Modelleigenschaft fur $S5$ nicht anwendbar ist, weil die Formeln $\phi_n$ in solchen Rahmen nicht erfüllbar sind.}

Sei $\kripke{M}$ ein Modell über S5 und $x \in M$ ein Punkt, der $\phi_1$ erfüllt. Aufgrund der ersten Konjunktion in
$\phi_1$ für $i = 0$ muss es $y, z \in M$ geben mit $xRy$ und $xRz$, wobei o.E. $\kripke{M}, y \vDash p_1$ sowie
$\kripke{M}, z \vDash \neg p_1$. Aus der Symmetrie der Relation $R$ folgt jedoch $yRx$ und wegen Transitivität auch
$yRz$. Die zweite Konjunktion fordert jedoch für $i = 1$, dass jeder Nachfolger von $y$ (somit auch $z$) $p_1$ erfüllen
muss. Dies führt zu einem Widerspruch, $\kripke{M}, x \not\vDash \phi_1$.

Die oben geschilderten Bedingungen müssen auch für alle $n$ mit $n > 1$ gelten. Folglich ist kein $\phi_n$, $n > 1$ über
S5 erfüllbar.

Eine Ausnahme bildet $\phi_0$, da hier die zweite Konjunktion leer ist.

\aufgabe{4}{Gegenseitige Simulation}
\angabe{Zeigen Sie, dass zwischen den Wurzeln der beiden Modelle in beiden Richtungen Simulationen existieren,
aber keine Bisiumulation.}

\begin{center}
\begin{minipage}{0.4\textwidth}

\begin{tikzpicture}[->,>=stealth',shorten >=5pt, shorten <=5pt,auto,node distance=1.5cm,
  main node/.style={circle,inner sep=2pt,minimum size=1pt,fill,=black,draw,font=\sffamily\Large\bfseries},
  bnode/.style={circle,inner sep=2pt,minimum size=1pt,fill=black,draw,font=\sffamily\Large\bfseries,label=east:B},
  cnode/.style={circle,inner sep=2pt,minimum size=1pt,fill=black,draw,font=\sffamily\Large\bfseries,label=east:C},
  invnode/.style={circle,inner sep=2pt,minimum size=1pt,draw=white,font=\sffamily\Large\bfseries}
  ]

  \node[main node,label=west:$x_1$] (1) {};
  \node[invnode] (3) [below of=1] {};
  \node[main node,label=west:$x_{21}$] (2) [left of=3] {};
  \node[main node,label=west:$x_{22}$] (4) [right of=3] {};
  \node[bnode,label=west:$x_{31}$] (5) [below of=2] {};
  \node[cnode,label=west:$x_{32}$] (6) [below of=3] {};
  \node[cnode,label=west:$x_{33}$] (7) [below of=4] {};

  % fuer einzelne pfeile, die unterschiedlich sind: edge [<->] node (zb)
  \path[every node/.style={font=\sffamily\small}]
	(1) edge node [right] {} (2)
		edge node [right] {} (4)
	(2) edge [<->] node [left] {} (5)
		edge [out=-20,in=110] node [right] {} (6)
	(4) edge [<->] node [left] {} (7)
	(6) edge [out=150,in=-60] node [left] {} (2);
\end{tikzpicture}
\end{minipage}
\begin{minipage}{0.4\textwidth}
\begin{tikzpicture}[->,>=stealth',shorten >=5pt, shorten <=5pt,auto,node distance=1.5cm,
  main node/.style={circle,inner sep=2pt,minimum size=1pt,fill,=black,draw,font=\sffamily\Large\bfseries},
  bnode/.style={circle,inner sep=2pt,minimum size=1pt,fill=black,draw,font=\sffamily\Large\bfseries,label=east:B},
  cnode/.style={circle,inner sep=2pt,minimum size=1pt,fill=black,draw,font=\sffamily\Large\bfseries,label=east:C},
  invnode/.style={circle,inner sep=2pt,minimum size=1pt,draw=white,font=\sffamily\Large\bfseries}
  ]


  \node[main node,label=west:$y_1$] (1) {};
  \node[invnode] (3) [below of=1] {};
  \node[main node,label=west:$y_{21}$] (2) [left of=3] {};
  \node[main node,label=west:$y_{22}$] (4) [right of=3] {};
  \node[bnode,label=west:$y_{31}$] (5) [below of=2] {};
  \node[cnode,label=west:$y_{32}$] (6) [below of=3] {};
  \node[bnode,label=west:$y_{33}$] (7) [below of=4] {};

  \path[every node/.style={font=\sffamily\small}]
	(1) edge node [right] {} (2)
		edge node [right] {} (4)
	(2) edge [<->] node [left] {} (5)
		edge [out=-20,in=110] node [right] {} (6)
	(4) edge [<->] node [left] {} (7)
	(6) edge [out=150,in=-60] node [left] {} (2);
\end{tikzpicture}
\end{minipage}
\end{center}

Für die Existenz von Simulationen sowie die Nichtexistenz von Bisimulation reicht es,
lediglich Beispiele anzugeben.

\begin{enumerate}
	\item Simulation von links nach rechts: $S = \{ (x_1,y_1), (x_{21}, y_{21}), (x_{22}, y_{21}), (x_{31}, y_{31}), (x_{32},  y_{32}), (x_{33}, y_{32}) \}$
		Es ist leicht nachzuprüfen, dass dies eine Simulation ist.
	\item Simulation von rechts nach links: $S = \{ (y_1,x_1), (y_{21}, x_{21}), (y_{22}, x_{21}), (y_{31}, x_{31}), (y_{32},  x_{32}), (y_{33}, x_{31}) \}$
		Wiederum ist leicht zu sehen, dass dies eine Simulation ist.
	\item Bisimulation: Die Formel $\Box \dia B$ wird an der Wurzel im rechten Modell erfüllt, nicht jedoch im linken Modell.
\end{enumerate}

%Gegeben sind die beiden baumförmigen Kripkemodelle \kripke{M} (links) und
%\kripke{N} (rechts).

%$\mathrm{Z\kern-0.4em\raise-0.4ex\hbox{Z}}$: Zwischen den Wurzeln ($\{x,y\}_1$)
%der beiden Modelle existiert in beiden Richtungen Simulationen, aber keine
%Bisimulation.

%Diese simulieren sich gegenseitig, wenn es sowohl eine Relation $S \subseteq M \times
%N$ und eine Relation $R \subseteq N \times M, (S \neq R)$ gibt und beide Relationen
%Simulationen sind.

%Zunächst zeigen wir, dass eine solche Relation $S(R)$ nur existieren kann, wenn
%$x_1 S(R) y_1$ gilt. Gilt nämlich oBdA. $x_1 S(R) y_i, i \neq 1$ dann folgt durch
%Betrachtung der Nachfolger, wie nach Definitionsteil (ii), dass auf der rechten
%Seite nicht ausrechend Nachfolger vorhanden sind.

%Beweisführung durch Überprüfung der Definition von Simulation:

%Nicht Modal:

%Es gilt: $\kripke{M}, x_1 \vDash P$ und $\kripke{N},y_1 \vDash P$ (i). Für alle
%Nachfolger von $x_1$, $x_2$, muss nun gelten: $\exists w.y_1 R w \wedge x_2 S w$
%(ii).
%\begin{enumerate}
	%\item $w = y_{21}$ erfüllt diese Bedingung, da in beiden Punkten $P$ gilt.
		%Überprüfe nun erneut (ii) bei dem Nachfolger $x_{32}$ von $x_2$. Nach
		%Definition muss wiederum gelten: $\exists w'.y_{21} R w' \wedge x_{32} S
		%w'$. Im Modell \kripke{N} kann nur $w' = y_{31}$ gelten. Damit ist $S$
		%keine Simulation, da $\kripke{M}, x_{32} \nvDash P$ und $\kripke{N},
		%y_{32} \vDash P$
	%\item Für $w = y_{22}$ ist der Beweis analog bei Wahl Nachfolger $x_{31}$
		%von $x_2$.
%\end{enumerate}

%Im modalen Fall erfolgt der Beweis analog mithilfe des
%Bisimulationsinvarianzsatzes, wenn $P$ eine modale Formel ist.


\stepcounter{blatt}
% vim: set filetype=tex:

\aufgabe{1}{Simulation verkehrt}
\angabe{%
	Man definiere eine Rückwärtssimulation zwischen Modellen $\kripke{M}_1 =
	(M_1, R_1, V_1)$ und $\kripke{M}_2 = (M_2, R_2, V_2)$ als eine binäre
	Relation $S \subseteq M_1 \times M_2$, so dass für alle $x, y$ mit $xSy$
	gilt
	\begin{enumerate}
		\item für jedes Atom $p$ mit $x \in V_1(p)$ gilt $y \in V_2(p)$
		\item für jedes $y'$ mit $yR_2y'$ existiert $x'$ mit $xR_1x'$ und
			$x'Sy'$.
	\end{enumerate}
	Zeigen Sie, dass, wenn $S$ eine Rückwärtssimulation ist und $\phi$ eine
	Formel in der durch die Grammatik
	\begin{center}
		$ \phi \coloncolonequals p~|~\top~|~\phi_1 \wedge \phi_2~|~\phi_1 \vee
	\phi_2~|~\Box \phi $
	\end{center}
	definierten Menge modaler Formeln, dann $S$ Erfülltheit von $\phi$ bewahrt,
	d.h. aus $xSy$ und $\kripke{M}_1, x \vDash \phi$ folgt $\kripke{M}_2, y
	\vDash \phi$.
}







\newcommand{\ex}{\exists R.}
\newcommand{\and}{\sqcap}

\aufgabe{2}{Subsumption in $\el$}
\angabe{%
	Wenden Sie den Subsumptionsalgorithmus für $\el$ an, um zu entscheiden, ob
	folgende Subsumptionen gültig sind:
\begin{equation}
	A \and B \and \ex (B \and C) \and \ex ( B \and C
	\and \ex D)
	\sqsubseteq
	A \and \ex (B \and \ex D) \and \ex ( C \and \ex D)
\end{equation}

\begin{equation}
	A \and \ex ( B \and \ex C) \and \ex (B \and \ex D)
	\sqsubseteq
	A \and \ex (B \and \ex (C \and D))
\end{equation}
}

Da jede Kante im Beschreibungsbaum die Relation $R$ darstellt, wird sie der
Einfachkeit halber weggelassen. \\
Um zu zeigen, ob eine Subsumption gültig ist, reicht es zu zeigen, dass es
eine Simulation von der rechten auf die linke Seite des Subsumptionszeichens
gibt.

\begin{center}
\begin{minipage}{0.4\textwidth}
\begin{tikzpicture}
	[edge from parent/.style={->,draw}]
	\node{A,B}
		child{node{B,C}}
		child{node{B,C}
			child {node{D}}
		};
\end{tikzpicture}
\end{minipage}
\begin{minipage}{0.4\textwidth}
\begin{tikzpicture}
	[edge from parent/.style={->,draw}]
	\node{A}
		child{node{B}
			child{node{D}}}
		child{node{C}
			child{node{D}}
		};
\end{tikzpicture}
\end{minipage}
\end{center}

An der Wurzel des rechten Beschreibungsbaums ist nur $A$ erfüllt. Links ist
zusätzlich noch $B$ erfüllt. Der erste Schritt ist also schon getan. Als
nächstes kann man links jeden Pfad der rechten Seite abdecken, indem man links
stets den rechten Zweig wählt. Nach dem ersten Schritt werden im linken
Beschreibungsbaum sowohl $B$ als auch $C$ erfüllt. Damit ist es egal in welche
Richtung man im rechten Baum geht, da hier nur eines der beiden Labels erfüllt
sein muss. Und auch der letzte Schritt zu $D$ ist mit dem rechten Zweig im
linken Baum zu simulieren. Damit existiert eine Simulation von rechts nach links
und die Subsumption ist gültig.

\begin{center}
\begin{minipage}{0.4\textwidth}
\begin{tikzpicture}
	[edge from parent/.style={->,draw}]
	\node{A}
		child{node{B}
			child{node{C}}}
		child{node{B}
			child{node{D}}
		};
\end{tikzpicture}
\end{minipage}
\begin{minipage}{0.4\textwidth}
\begin{tikzpicture}
	[edge from parent/.style={->,draw}]
	\node{A}
		child{node{B}
			child{node{C,D}}
		};
\end{tikzpicture}
\end{minipage}
\end{center}

In beiden Bäumen lässt sich der erste Schritt machen. Allerdings ist der zweite
Schritt im linken Baum unter dem Gesichtspunkt einer Simulation nicht möglich,
da entweder nur $C$ oder $D$ erfüllt werden. Der rechte Beschreibungsbaum
fordert jedoch, dass $C$ und $D$ erfüllt sein müssen. Damit ist diese
Subsumption ungültig.

\end{document}
