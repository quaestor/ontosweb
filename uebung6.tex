% vim: set filetype=tex:

\aufgabe{1}{Simulation verkehrt}
\angabe{%
	Man definiere eine Rückwärtssimulation zwischen Modellen $\kripke{M}_1 =
	(M_1, R_1, V_1)$ und $\kripke{M}_2 = (M_2, R_2, V_2)$ als eine binäre
	Relation $S \subseteq M_1 \times M_2$, so dass für alle $x, y$ mit $xSy$
	gilt
	\begin{enumerate}
		\item für jedes Atom $p$ mit $x \in V_1(p)$ gilt $y \in V_2(p)$
		\item für jedes $y'$ mit $yR_2y'$ existiert $x'$ mit $xR_1x'$ und
			$x'Sy'$.
	\end{enumerate}
	Zeigen Sie, dass, wenn $S$ eine Rückwärtssimulation ist und $\phi$ eine
	Formel in der durch die Grammatik
	\begin{center}
		$ \phi \coloncolonequals p~|~\top~|~\phi_1 \wedge \phi_2~|~\phi_1 \vee
	\phi_2~|~\Box \phi $
	\end{center}
	definierten Menge modaler Formeln, dann $S$ Erfülltheit von $\phi$ bewahrt,
	d.h. aus $xSy$ und $\kripke{M}_1, x \vDash \phi$ folgt $\kripke{M}_2, y
	\vDash \phi$.
}







\newcommand{\ex}{\exists R.}
\newcommand{\and}{\sqcap}

\aufgabe{2}{Subsumption in \el}
\angabe{%
	Wenden Sie den Subsumptionsalgorithmus für \el~ an, um zu entscheiden, ob
	folgende Subsumptionen gültig sind:
\begin{equation}
	A \and B \and \ex (B \and C) \and \ex ( B \and C
	\and \ex D)
	\sqsubseteq
	A \and \ex (B \and \ex D) \and \ex ( C \and \ex D)
\end{equation}

\begin{equation}
	A \and \ex ( B \and \ex C) \and \ex (B \and \ex D)
	\sqsubseteq
	A \and \ex (B \and \ex (C \and D))
\end{equation}
}

Da jede Kante im Beschreibungsbaum die Relation $R$ darstellt, wird sie der
Einfachkeit halber weggelassen. \\
Um zu zeigen, dass diese Subsumptionen gültig sind, reicht es zu zeigen, dass es
eine Simulation von $y_1$ nach $x_1$ gibt.


\begin{center}
\begin{minipage}{0.4\textwidth}
\begin{tikzpicture}[->,>=stealth',shorten >=5pt, shorten <=5pt,auto,node distance=1.5cm,
  main node/.style={circle,inner sep=2pt,minimum size=1pt,fill,=black,draw,font=\sffamily\Large\bfseries},
  not main node/.style={circle,inner sep=2pt,minimum size=1pt,fill=black,draw,font=\sffamily\Large\bfseries}]

  \node[main node,label=west:$x_1$,label=east:{A}] (1) {};
  \node[main node,label=west:$x_{21}$,label=east:{B,C}] (2) [below left of=1] {};
  \node[main node,label=west:$x_{22}$,label=east:{B,C}] (4) [below right of=1] {};
  \node[not main node,label=west:$x_{32}$,label=east:{D}] (5) [below of=4] {};

  \path[every node/.style={font=\sffamily\small}]
    (1) edge node [right] {} (2)
		edge node [left]  {} (4)
	(4) edge node [right] {} (5);
\end{tikzpicture}
\end{minipage}
\begin{minipage}{0.4\textwidth}
\begin{tikzpicture}[->,>=stealth',shorten >=5pt, shorten <=5pt,auto,node distance=1.5cm,
  main node/.style={circle,inner sep=2pt,minimum
  size=1pt,fill,=black,draw,font=\sffamily\Large\bfseries},
  not main node/.style={circle,inner sep=2pt,minimum size=1pt,fill=black,draw,font=\sffamily\Large\bfseries}]

  \node[main node,label=west:$y_1$,label=east:{A}] (1) {};
  \node[main node,label=west:$y_{21}$,label=east:{B}] (2) [below left of=1] {};
  \node[main node,label=west:$y_{31}$,label=east:{D}] (3) [below of=2] {};
  \node[main node,label=west:$y_{22}$,label=east:{C}] (4) [below right of=1] {};
  \node[not main node,label=west:$y_{32}$,label=east:{D}] (5) [below of=4] {};

  \path[every node/.style={font=\sffamily\small}]
    (1) edge node [right] {} (2)
		edge node [left]  {} (4)
	(2) edge node [right] {} (3)
	(4) edge node [right] {} (5);
\end{tikzpicture}
\end{minipage}
\end{center}

Die gültigen Simulationen sind $S_1 = \{(y_1 R x_1), (y_{21}, x_{22}), (y_{31},
x_{32})\}$
und $S_2 = \{(y_1, x_1),\\ (y_{22}, x_{22}), (y_{32}, x_{32})\}$.


\begin{center}
\begin{minipage}{0.4\textwidth}
\begin{tikzpicture}[->,>=stealth',shorten >=5pt, shorten <=5pt,auto,node distance=1.5cm,
  main node/.style={circle,inner sep=2pt,minimum
  size=1pt,fill,=black,draw,font=\sffamily\Large\bfseries},
  not main node/.style={circle,inner sep=2pt,minimum size=1pt,fill=black,draw,font=\sffamily\Large\bfseries}]

  \node[main node,label=west:$x_1$,label=east:{A}] (1) {};
  \node[main node,label=west:$x_{21}$,label=east:{B}] (2) [below left of=1] {};
  \node[main node,label=west:$x_{31}$,label=east:{C}] (3) [below of=2] {};
  \node[main node,label=west:$x_{22}$,label=east:{B}] (4) [below right of=1] {};
  \node[not main node,label=west:$x_{32}$,label=east:{D}] (5) [below of=4] {};

  \path[every node/.style={font=\sffamily\small}]
    (1) edge node [right] {} (2)
		edge node [left]  {} (4)
	(2) edge node [right] {} (3)
	(4) edge node [right] {} (5);
\end{tikzpicture}
\end{minipage}
\begin{minipage}{0.4\textwidth}
\begin{tikzpicture}[->,>=stealth',shorten >=5pt, shorten <=5pt,auto,node distance=1.5cm,
  main node/.style={circle,inner sep=2pt,minimum
  size=1pt,fill,=black,draw,font=\sffamily\Large\bfseries},
  not main node/.style={circle,inner sep=2pt,minimum size=1pt,fill=black,draw,font=\sffamily\Large\bfseries}]

  \node[main node,label=west:$y_1$,label=east:{A}] (1) {};
  \node[main node,label=west:$y_{21}$,label=east:{B}] (2) [below of=1] {};
  \node[main node,label=west:$y_{22}$,label=east:{C,D}] (4) [below of=2] {};

  \path[every node/.style={font=\sffamily\small}]
    (1) edge node [right] {} (2)
	(2) edge node [right] {} (4);
\end{tikzpicture}
\end{minipage}
\end{center}

Die beiden möglichen Simulationen $S_1 = \{(y_1, x_1), (y_{21}, x_{21}), (y_{22},
x_{31})\}$ und $S_2 = \{ (y_1,  x_1),\\ (y_{21}, x_{22}), (y_{22},
x_{32})\}$ sind ungültig,
da sowohl an $x_{31}$ als auch an $x_{32}$ C,D gelten muss, damit die
Simulation gültig wird.
