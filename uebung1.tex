\aufgabe{1}{Tableaux-Algorithmus}
\angabe{%
Wenden Sie den Tableaux-Algorithmus auf die Formel
\[
  \left( (A \wedge B) \to \neg C \right) \wedge (\neg A \to C) \wedge (\neg B
  \to C) \wedge \neg (B \to \neg C)
\]
an. Ist die Formel erfüllbar? [Hinweis: Sie müssen zunächst alle Konnektive
durch $\neg$ und $\wedge$ dekodieren.]
}

\vspace{-2em}
\begin{align*}
  \varphi & = \left( (A \wedge B) \to \neg C \right) \wedge (\neg A \to C) \wedge (\neg B \to C) \wedge
        \neg (B \to \neg C)
\intertext{… Auflösen der Implikationen …}
    & = \left( \neg (A \wedge B) \vee \neg C \right) \wedge (\neg \neg A \vee C) \wedge (\neg \neg B \vee C)
        \wedge \neg (\neg B \vee \neg C) \\ 
\intertext{… und der Disjunktionen sowie der doppelten Negationen …}
    & = \neg \left( A \wedge B \wedge C \right) \wedge \neg (\neg A \wedge \neg C) \wedge \neg (\neg B \wedge \neg C)
        \wedge (B \wedge C)
\end{align*}
Aus der ersten und letzten Klammer ergibt sich in dieser Form offensichtlich, dass für eine erfüllende Belegung $A =
\bot, B = \top, C = \top$ gelten muss. Nach Überprüfung der restlichen Terme erhält man kurzerhand das Ergebnis, dass
\[
  \kappa : \mathcal{A} \to \mathbf{2} : A \mapsto \bot, \ B \mapsto \top, \ C \mapsto \top
\]
die Formel erfüllt. Der Tableaux-Algorithmus sollte uns im Folgenden also dasselbe berichten:

\begin{gather*}
    \big\{ \neg \left( A \wedge B \wedge C \right) \cm{\wedge} \neg (\neg A \wedge \neg C) \cm{\wedge} \neg (\neg B
          \wedge \neg C) \cm{\wedge} B \cm{\wedge} C \big\} \\
\intertext{… Regel $(\wedge)$ für alle $\wedge$-Konnektive auf oberster Ebene …}
    \big\{ \cm{\neg} \left( \cm{A} \wedge (B \wedge C) \right), \neg (\neg A \wedge \neg C), \neg (\neg B \wedge \neg
          C), B, C \big\} \\
\intertext{… Regel $(\neg\wedge)$ auf erstes Element (linke Seite gewählt) …}
    \big\{ \neg A, \cm{\neg} (\neg A \wedge \cm{\neg C}), \neg (\neg B \wedge \neg C), B, C \big\} \\
\intertext{… Regel $(\neg\wedge)$ auf zweites Element (rechte Seite gewählt) …}
    \big\{ \neg A, \cm{\neg \neg C}, \neg (\neg B \wedge \neg C), B, C \big\} \\
\intertext{… Regel $(\neg\neg)$ auf zweites Element …}
    \big\{ \neg A, C, \cm{\neg} (\cm{\neg B} \wedge \neg C), B, C \big\} \\
\intertext{… Regel $(\neg\wedge)$ auf drittes Element (linke Seite gewählt) …}
    \big\{ \neg A, C, \cm{\neg \neg B}, B \big\} \\
\intertext{… Regel $(\neg\neg)$ auf drittes Element …}
    \big\{ \neg A, C, B \big\}
\end{gather*}
Hier ist keine Regel mehr anwendbar, die Formel wird wie erwartet als erfüllbar erkannt mit oben beschriebener
Wahrheitsbelegung.
\aufgabe{2}{Resolution}
\angabe{%
Wenden Sie den Resolutionsalgorithmus auf die CNF
\[
  \big\{ \{\neg D, A, B\}, \{C, B, \neg A\}, \{\neg D, B\}, \{\neg B, A\}, \{D,
  C\}, \{\neg C, B, \neg A\}, \{D, B, \neg C\}, \{\neg B, \neg A\} \big\}
\]
an -- ist die CNF erfüllbar? Wenden Sie alternativ DPLL an -- ergeben sich Vereinfachungen?
}
\vspace{-2em}
\[
 \varphi = \big\{ \{\neg D, A, B\}, \{C, B, \neg A\}, \{\neg D, B\}, \{\neg B, A\}, \{D, C\}, \{\neg C, B, \neg A\},
            \{D, B, \neg C\}, \{\neg B, \neg A\} \big\}
\]
Aus der vierten und der letzten Klausel lässt sich sofort
\begin{prooftree}
    \AxiomC{$\{\neg B, \cm{A}\}$}
    \AxiomC{$\{\neg B, \cm{\neg A}\}$}
    \BinaryInfC{$\{\neg B\}$}
\end{prooftree}
resolvieren. Ebenso erhält man durch die dritte und fünfte Klausel
\begin{prooftree}
        \AxiomC{$\{\cm{\neg D}, B\}$}
        \AxiomC{$\{\cm{D}, C\}$}
    \BinaryInfC{$\{B, C\}$}
\end{prooftree}
Desweiteren kann man durch eine Folge von Anwendungen der Resolutionsregel folgende Klauseln erzeugen:
\begin{prooftree}
        \AxiomC{$\{\cm{\neg D}, A, B\}$}
        \AxiomC{$\{\cm{D}, B, \neg C\}$}
    \RightLabel{\scriptsize{1 und 7}}
    \BinaryInfC{$\{A, B, \neg C\}$}
\end{prooftree}
sowie
\begin{prooftree}
    \AxiomC{$\{\neg D, \cm{A}, B\}$}
    \AxiomC{$\{\neg C, B, \cm{\neg A}\}$}
    \RightLabel{\scriptsize{1 und 6}}
    \BinaryInfC{$\{\neg D, B, \neg C\}$}
    \AxiomC{$\{D, \cm{B}, \neg C\}$}
    \AxiomC{$\{\cm{\neg B}, \neg A\}$}
    \RightLabel{\scriptsize{7 und 8}}
    \BinaryInfC{$\{D, \neg C, \neg A\}$}
\BinaryInfC{$\{\neg A, B, \neg C\}$}
\end{prooftree}
Resolviert man diese nun in umgekehrter Reihenfolge, so ergibt sich die leere Klausel. Die Formel ist folglich
unerfüllbar.

\vspace{2em}

Nun soll noch der Algorithmus DPLL auf dieselbe Formel angewendet werden. Dazu wählen wir zuerst die Variable $B$. Im
Falle $B = \top$ wird $\varphi / B$ untersucht:
\[
  \varphi / B = \big\{ \{\cm{A}\}, \{D, C\}, \{\cm{\neg A}\} \big\}
\]
Anwenden der Optimierungsmethode \emph{Unit Propagation} ($\{A\} \in \varphi \Rightarrow \varphi := \varphi / A$)
liefert sofort die leere Klausel. Es muss also mit \emph{Backtracking} der zweite Fall, $B = \bot$, untersucht werden:
\[
  \varphi := \varphi / \neg B = \big\{ \{\neg D, A\}, \{C, \neg A\}, \{\cm{\neg D}\}, \{D, C\}, \{\neg C, \neg A\},
  \{D, \neg C\}, \{\neg A\} \big\}
\]
Der nächste Schritt im Algorithmus ist erneut die \emph{Unit Propagation}, dieses Mal mit dem Literal $\neg D$
\[
  \varphi := \varphi / \neg D = \big\{ \{C, \neg A\}, \{\cm{C}\}, \{\neg C, \neg A\}, \{\cm{\neg C}\}, \{\neg A\} \big\}
\]
Nach einer weiteren \emph{Unit Propagation} ergibt sich auch hier die leere Klausel. Der Algorithmus bricht ab, die
Formel ist nicht erfüllbar.

\aufgabe{3}{Maximale konsistente CNF}
\angabe{%
Man erinnere sich, dass eine Klauselmenge (also eine CNF) $\phi$
\emph{resolutionsabgeschlossen} heißt, wenn aus $C \cup \{A\}, D \cup \{\neg
A\} \in \phi$ stets $C \cup D \in \phi$ folgt. Wir nennen eine CNF $\phi$ \emph{konsistent}, wenn sich aus ihr durch wiederholtes Anwenden der Resolutionsregel \emph{nicht} die leere Klausel herleiten lässt, wenn sie also vom Resolutionsalgorithmus als erfüllbar erkannt wird. Gegeben eine endliche Menge $\mathcal{A}$ von Aussagenvariablen heißt $\phi$ eine \emph{CNF über $\mathcal{A}$}, wenn $\phi$ nur Aussagenvariablen aus $\mathcal{A}$ erwähnt, und \emph{maximal konsistent über $\mathcal{A}$}, wenn $\phi$ maximal unter den konsistenten CNF über $\mathcal{A}$ ist. Beweisen Sie ohne Verwendung des Vollständigkeitsbeweises aus der Vorlesung folgende Aussagen:
\begin{enumerate}
	\item Jede konsistente CNF über $\mathcal{A}$ ist enthalten in einer maximal konsistenten CNF über $\mathcal{A}$.
	\item Sei $\phi$ eine maximal konsistente CNF über $\mathcal{A}$.
		\begin{enumerate}
			\item Für $A \in \mathcal{A}$ gilt $\{A\} \not\in \phi$ genau dann, wenn $\{\neg A\} \in \phi$.
			\item $\phi$ ist resolutionsabgeschlossen.
			\item $\phi$ ist abgeschlossen unter Abschwächung von Klauseln, d.h. aus $C \in \phi$ und $C \subseteq D$ folgt $D \in \phi$.
			\item Für Klauseln $C, D$ gilt $C \cup D \in \phi$ genau dann, wenn $C \in \phi$ oder $D \in \phi$.
			\item Die durch
				\[ \kappa(A) = \top \Leftrightarrow \{A\} \in \phi \]
				  gegebene Wahrheitsbelegung $\kappa\ :\ \mathcal{A} \to \mathbf{2}$ erfüllt $\phi$.
		\end{enumerate}
	\item Der Resolutionsalgorithmus ist vollständig, d.h.\ jede konsistente CNF ist erfüllbar.
\end{enumerate}
}
\begin{enumerate}
    \item Da Maximalität von konsistenten CNF über die Teilmengenrelation definiert ist, also 
          \[
            \phi \text{ maximal konsistent } \Leftrightarrow \forall \psi\ .\ \phi \subseteq \psi \Rightarrow \psi = \phi,
          \]
          ist jede konsistente CNF Teilmenge einer maximal konsistenten CNF: Die Menge der konsistenten CNF ist
          endlich und da auf den konsistenten CNF mittels der Teilmengenrelation eine partielle Ordnung definiert
          wird, existiert für alle Teilmengen von miteinander vergleichbaren konsistenten CNF (im Sinne der
          Teilmengenrelation) ein Maximum.
    \item
        \begin{enumerate}
                \item Hier gilt es zum einen, zu zeigen, dass \emph{jedes} Literal $A$ aus $\mathcal{A}$ (entweder
                positiv oder negativ) als Klausel $\{A\}$ oder $\{\neg A\}$ in $\phi$ vorkommt und zum anderen, dass
                nicht beide Klauseln zugleich vorkommen können:

                Annahme: $\{A\}, \{\neg A\} \not\in \phi$ und $\phi$ maximal. Wird nun o.B.d.A.\ $\{A\}$ zur
                Klauselmenge $\phi$ hinzugefügt, so ändert dies nichts an der Konsistenz der CNF, da für Klauseln C
                folgt:
                \begin{itemize}
                        \item[]$A \in C$: Resolution mit $\{A\}$ nicht möglich
                        \item[]$\neg A \in C$: Resolution mit $\{A\}$ liefert $D := C \setminus \{\neg A\}$.
                                Angenommen, $D$ ließe sich nun zur leeren Klausel resolvieren, dann ließe sich $C$ zu
                                $\{\neg A\}$ resolvieren, was im Widerspruch zur Annahme steht. Daher muss $\phi$ unter
                                Hinzunahme von $D$ und $\{A\}$ konsistent bleiben.
                        \item[]$A, \neg A \not \in C$: Resolution mit $\{A\}$ nicht möglich.
                \end{itemize}
                Da in jedem dieser Fälle $\{A\}$ hinzugefügt werden konnte, kann $\phi$ nicht maximal sein. Es muss
                also $\{A\}$ oder $\{\neg A\}$ in $\phi$ enthalten sein.

                Annahme: $\{A\} \in \phi, \{\neg A\} \in \phi$: dann lassen sich diese beiden Klauseln zur leeren
                Klausel resolvieren und $\phi$ ist nicht konsistent.

                Damit gilt die Behauptung.

                \item Sei $\phi$ maximal konsistent und nicht resolutionsabgeschlossen. Dann kann durch Resolution eine
                weitere Klausel hinzugefügt werden, was der Annahme widerspricht, dass $\phi$ maximal ist.

                \item Erweitert man eine Klausel $C \in \phi$ um eine beliebige Menge von Literalen $X$, so gilt: da $C$
                sich nicht auf die leere Klausel resolvieren lässt (Konsistenz), kann $C \cup X$ nicht auf $X$
                resolviert werden, da immer ein Teil von $C$ bei der Resolution übrig bleibt. Damit kann $C \cup X$
                auch nicht zur leeren Klausel resolviert werden und auch keine andere Klausel durch Resolution mit $C
                \cup X$ und $\phi \cup \{C \cup X\}$ ist konsistent.

                \item Die Richtung \glqq$\Leftarrow$\grqq\ sofort aus c), da sowohl $C$ als auch $D$ Teilmengen von $C
                \cup D$ sind.

                \glqq$\Rightarrow$\grqq: angenommen $C \cup D \in \phi, C \not\in \phi, D \not\in \phi,\ \phi$ maximal
                konsistent. Dann ist $\phi\,\cup \{C\}$ nicht konsistent, d.h. $C$ führt durch Resolution zur leeren
                Menge. Dies bedeutet jedoch für $\phi$, dass $C \cup D$ dort zu $(C \cup D) \setminus C$ (also einer
                Teilmenge von $D$) resolviert werden kann. Nach c) ist damit $D$ in $\phi$, was im Widerspruch zur
                Annahme steht.

                \item Nach a) gilt für jedes $A \in \mathcal{A}$ entweder $\{A\} \in \phi$ oder $\{\neg A\} \in \phi$.
                Die gegebene Belegung erfüllt all diese einelementigen Klauseln. Nach d) kann jede Klausel aus $\phi$
                zerlegt werden in $C = D \cup E$, wobei mindestens eine der beiden Teilklauseln in $\phi$ ist. Durch
                wiederholte Anwendung kann man so jede Klausel $C$ in eine Vereinigung von einelementigen Klauseln
                unterteilen, von denen mindestens eine in $\phi$ und damit durch $\kappa$ erfüllbar ist und damit auch
                $C$.
        \end{enumerate}
    \item Nach 2. e) existiert für jede maximale konsistente CNF eine erfüllende Belegung und nach 1. ist jede
    konsistente CNF $\psi$ eine Teilmenge einer geeigneten maximalen konsistenten CNF $\phi$. Damit gilt $\kappa_{\psi}
    = \kappa_{\phi}$, da $\phi$ alle Klauseln von $\psi$ enthält.
\end{enumerate}
\vfill
\begin{figure}[!ht]
\centering
%\includegraphics[height=16cm]{ontorab.jpg}\\
{\small Ontology of a Rabbit \textcopyright 2012 Kit Lang}
\end{figure}
\vfill
