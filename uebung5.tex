% vim: set filetype=tex:
\newcommand{\defin}[2]{#1 $\sqsubseteq$ #2}
\newcommand{\ando}{$\sqcap$\hspace{1pt}}
\newcommand{\kripke}[1]{\ensuremath{\mathcal{#1}}}
\newcommand{\relation}[1]{\ensuremath{\mathcal{#1}}}
\newcommand{\dia}{\lozenge}

\aufgabe{1}{Modellierung terminologischen Wissens}

\angabe{%
Verwenden Sie den Tableaukalkül, um folgende Formeln auf Erfüllbarkeit zu prüfen:
\begin{equation*}
\Box(\dia p) \wedge \Box(p \rightarrow{} (\dia q \vee \dia r)) \wedge
\dia p \wedge \dia \neg p \wedge \Box \Box ((q \vee r) \rightarrow{} \neg p)
\wedge \Box (\neg p \rightarrow{} \Box \neg p)
\label{erfbar1}
\end{equation*}
\begin{equation*}
	\Box(p \rightarrow{} (\dia q \vee \dia r)) \wedge (\dia p \vee \dia \neg p)
	\wedge \Box(\neg p \rightarrow{} \Box \neg p) \wedge \Box \Box (p \wedge
	\neg r)
	\label{erfbar2}
\end{equation*}
}

Anwendung der Tableau-Regeln für modale Formeln auf modale Formeln.

\begin{prooftree}
\AxiomC{$\Box(\dia p) \wedge \Box(p \rightarrow{} (\dia q \vee \dia r)) \wedge
\dia p \wedge \dia \neg p \wedge \Box \Box ((q \vee r) \rightarrow{} \neg p)
\wedge \Box (\neg p \rightarrow{} \Box \neg p)$}

\UnaryInfC{$ \Box(\dia p) \wedge \Box(\neg p \vee (\dia q \vee \dia r)) \wedge
\dia p \wedge \dia \neg p \wedge \Box \Box (\neg(q \vee r) \vee \neg p)
\wedge \Box (\neg\neg p \vee \Box \neg p) $}

\UnaryInfC{$ \Box(\dia p) \wedge \Box(\neg p \vee \neg(\neg\dia q \wedge \neg\dia r)) \wedge
\dia p \wedge \dia \neg p \wedge \Box \Box (\neg\neg(\neg q \wedge \neg r) \vee \neg p)
\wedge \Box \neg(\neg\neg\neg p \wedge \Box \neg\neg p) $}

\UnaryInfC{$ \Box(\dia p) \wedge \Box\neg(\neg\neg p \wedge \neg\neg(\neg\dia q \wedge \neg\dia r)) \wedge
\dia p \wedge \dia \neg p \wedge \Box \Box \neg(\neg\neg\neg(\neg q \wedge \neg r)
\wedge \neg\neg p) \wedge \Box \neg(\neg\neg\neg p \wedge \Box \neg\neg p) $}

\UnaryInfC{$ \Box(\dia p) \wedge \Box\neg(p \wedge (\neg\dia q \wedge \neg\dia r)) \wedge
\dia p \wedge \dia \neg p \wedge \Box \Box \neg(\neg(\neg q \wedge \neg r)
\wedge  p) \wedge \Box \neg(\neg p \wedge \Box  p) $}

\UnaryInfC{$ \Box(\dia p) , \Box\neg(p \wedge (\neg\dia q \wedge \neg\dia r)) ,
\dia p , \dia \neg p , \Box \Box \neg(\neg(\neg q \wedge \neg r)
\wedge  p) , \Box \neg(\neg p \wedge \Box  p) $}

\UnaryInfC{$ \Box(\Box\neg p) , \Box\neg(p \wedge (\neg\Box\neg q \wedge
\neg\Box\neg r)) , \Box\neg p , \Box\neg \neg p , \Box \Box \neg(\neg(\neg q \wedge \neg r)
\wedge  p) , \Box \neg(\neg p \wedge \Box  p) $}

\UnaryInfC{$ (\Box\neg p) , \neg(p \wedge (\neg\Box\neg q \wedge
\neg\Box\neg r)) , \neg p , \neg \neg p ,  \Box \neg(\neg(\neg q \wedge \neg r)
\wedge  p) ,  \neg(\neg p \wedge \Box  p) $}

\UnaryInfC{$ (\Box\neg p) , \neg(p \wedge (\neg\Box\neg q \wedge
\neg\Box\neg r)) , \neg p , \neg \neg p ,  \Box \neg(\neg(\neg q \wedge \neg r)
\wedge  p) ,  \neg(\neg p \wedge \Box  p) $}



\end{prooftree}









\begin{center}
$ \Box(\neg p \vee (\dia q \vee \dia r)) \wedge (\dia p \vee \dia \neg p) \wedge
\Box (\neg \neg p \vee \Box \neg p) \wedge \Box \Box (p \wedge \neg r) $

$ \neg(\Box (\neg\neg p \wedge \neg(\dia q \vee \dia r)) \wedge (\neg\dia p
\wedge \neg\dia\neg p) \wedge \Box (\neg\neg\neg p \wedge \neg\Box\neg p) \wedge \Box\Box (p \wedge \neg r)) $

$ \neg(\Box (\neg\neg p \wedge (\neg\dia q \wedge \neg\dia r)) \wedge (\neg\dia p
\wedge \neg\dia\neg p) \wedge \Box (\neg\neg\neg p \wedge \neg\Box\neg p) \wedge \Box\Box (p \wedge \neg r)) $
\end{center}






\aufgabe{2}{Große Modelle}
\angabe{%
Gegeben seien Atome $q_1,q_2,\dots$. Man definiere die Familie von Formeln
$\phi_n$ durch
\begin{equation*}
\phi_n = \bigwedge^{n}_{i=0} \Box^i(\dia q_{i+1} \wedge \dia \neg
q_{i+1}) \wedge \displaystyle \bigwedge_{i=1}^{n} \Box^i(\Box^{\leq n - i}((q_i
\rightarrow \Box q_i) \wedge (\neg q_i \rightarrow{} \Box \neg q_i))).
\label{bigmodel}
\end{equation*}
Zeigen Sie, dass jedes Modell von $\phi_n$ mindestens $2^n$ Elemente hat. Geben
Sie eine polynomielle Abschätzung fúr die Größe von $\phi_n$ in Abhängigkeit
von $n$ an.
}

Sei $\kripke{M}$ ein Modell für $\phi_n$, $x\in M$ mit $\kripke{M}, x \vDash
\phi_n$.

Die erste Konjunktion in $\phi_n$ stellt sicher, dass jeder Punkt des
Modells, der von $x$ in maximal $n$ Schritten erreichbar ist, mindestens zwei
verschiedene Nachfolger besitzt.

Durch die zweite Konjunktion wird ein Atom $q_i$ bzw. seine Negation $\neg q_i$
von einem Punkt, der von $x$ ausgehend in $i$ Schritten auf alle seine Nachfolger propagiert.

\aufgabe{3}{Keine großen Modelle}

\aufgabe{4}{Gegenseitige Simulation}
\begin{center}
\begin{minipage}{0.4\textwidth}

\begin{tikzpicture}[->,>=stealth',shorten >=5pt, shorten <=5pt,auto,node distance=1.5cm,
  main node/.style={circle,inner sep=2pt,minimum size=1pt,fill,=black,draw,font=\sffamily\Large\bfseries},
  bnode/.style={circle,inner sep=2pt,minimum size=1pt,fill=black,draw,font=\sffamily\Large\bfseries,label=east:B},
  cnode/.style={circle,inner sep=2pt,minimum size=1pt,fill=black,draw,font=\sffamily\Large\bfseries,label=east:C},
  invnode/.style={circle,inner sep=2pt,minimum size=1pt,draw=white,font=\sffamily\Large\bfseries}
  ]

  \node[main node,label=west:$x_1$] (1) {};
  \node[invnode] (3) [below of=1] {};
  \node[main node,label=west:$x_{21}$] (2) [left of=3] {};
  \node[main node,label=west:$x_{22}$] (4) [right of=3] {};
  \node[bnode,label=west:$x_{31}$] (5) [below of=2] {};
  \node[cnode,label=west:$x_{32}$] (6) [below of=3] {};
  \node[cnode,label=west:$x_{33}$] (7) [below of=4] {};

  % fuer einzelne pfeile, die unterschiedlich sind: edge [<->] node (zb)
  \path[every node/.style={font=\sffamily\small}]
	(1) edge node [right] {} (2)
		edge node [right] {} (4)
	(2) edge [<->] node [left] {} (5)
		edge [out=-20,in=110] node [right] {} (6)
	(4) edge [<->] node [left] {} (7)
	(6) edge [out=150,in=-60] node [left] {} (2);
\end{tikzpicture}
\end{minipage}
\begin{minipage}{0.4\textwidth}
\begin{tikzpicture}[->,>=stealth',shorten >=5pt, shorten <=5pt,auto,node distance=1.5cm,
  main node/.style={circle,inner sep=2pt,minimum size=1pt,fill,=black,draw,font=\sffamily\Large\bfseries},
  bnode/.style={circle,inner sep=2pt,minimum size=1pt,fill=black,draw,font=\sffamily\Large\bfseries,label=east:B},
  cnode/.style={circle,inner sep=2pt,minimum size=1pt,fill=black,draw,font=\sffamily\Large\bfseries,label=east:C},
  invnode/.style={circle,inner sep=2pt,minimum size=1pt,draw=white,font=\sffamily\Large\bfseries}
  ]


  \node[main node,label=west:$y_1$] (1) {};
  \node[invnode] (3) [below of=1] {};
  \node[main node,label=west:$y_{21}$] (2) [left of=3] {};
  \node[main node,label=west:$y_{22}$] (4) [right of=3] {};
  \node[bnode,label=west:$y_{31}$] (5) [below of=2] {};
  \node[cnode,label=west:$y_{32}$] (6) [below of=3] {};
  \node[bnode,label=west:$y_{33}$] (7) [below of=4] {};

  \path[every node/.style={font=\sffamily\small}]
	(1) edge node [right] {} (2)
		edge node [right] {} (4)
	(2) edge [<->] node [left] {} (5)
		edge [out=-20,in=110] node [right] {} (6)
	(4) edge [<->] node [left] {} (7)
	(6) edge [out=150,in=-60] node [left] {} (2);
\end{tikzpicture}
\end{minipage}
\end{center}

Gegeben sind die beiden baumförmigen Kripkemodelle \kripke{M} (links) und
\kripke{N} (rechts).

$\mathrm{Z\kern-0.4em\raise-0.4ex\hbox{Z}}$: Zwischen den Wurzeln ($\{x,y\}_1$)
der beiden Modelle existiert in beiden Richtungen Simulationen, aber keine
Bisimulation.

Diese simulieren sich gegenseitig, wenn es sowohl eine Relation $S \subseteq M \times
N$ und eine Relation $R \subseteq N \times M, (S \neq R)$ gibt und beide Relationen
Simulationen sind.

Zunächst zeigen wir, dass eine solche Relation $S(R)$ nur existieren kann, wenn
$x_1 S(R) y_1$ gilt. Gilt nämlich oBdA. $x_1 S(R) y_i, i \neq 1$ dann folgt durch
Betrachtung der Nachfolger, wie nach Definitionsteil (ii), dass auf der rechten
Seite nicht ausrechend Nachfolger vorhanden sind.

Beweisführung durch Überprüfung der Definition von Simulation:

Nicht Modal:

Es gilt: $\kripke{M}, x_1 \vDash P$ und $\kripke{N},y_1 \vDash P$ (i). Für alle
Nachfolger von $x_1$, $x_2$, muss nun gelten: $\exists w.y_1 R w \wedge x_2 S w$
(ii).
\begin{enumerate}
	\item $w = y_{21}$ erfüllt diese Bedingung, da in beiden Punkten $P$ gilt.
		Überprüfe nun erneut (ii) bei dem Nachfolger $x_{32}$ von $x_2$. Nach
		Definition muss wiederum gelten: $\exists w'.y_{21} R w' \wedge x_{32} S
		w'$. Im Modell \kripke{N} kann nur $w' = y_{31}$ gelten. Damit ist $S$
		keine Simulation, da $\kripke{M}, x_{32} \nvDash P$ und $\kripke{N},
		y_{32} \vDash P$
	\item Für $w = y_{22}$ ist der Beweis analog bei Wahl Nachfolger $x_{31}$
		von $x_2$.
\end{enumerate}

Im modalen Fall erfolgt der Beweis analog mithilfe des
Bisimulationsinvarianzsatzes, wenn $P$ eine modale Formel ist.

