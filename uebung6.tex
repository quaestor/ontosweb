% vim: set filetype=tex:

\aufgabe{1}{Simulation verkehrt}
\angabe{%
	Man definiere eine Rückwärtssimulation zwischen Modellen $\kripke{M}_1 =
	(M_1, R_1, V_1)$ und $\kripke{M}_2 = (M_2, R_2, V_2)$ als eine binäre
	Relation $S \subseteq M_1 \times M_2$, so dass für alle $x, y$ mit $xSy$
	gilt
	\begin{enumerate}
		\item für jedes Atom $p$ mit $x \in V_1(p)$ gilt $y \in V_2(p)$
		\item für jedes $y'$ mit $yR_2y'$ existiert $x'$ mit $xR_1x'$ und
			$x'Sy'$.
	\end{enumerate}
	Zeigen Sie, dass, wenn $S$ eine Rückwärtssimulation ist und $\phi$ eine
	Formel in der durch die Grammatik
	\begin{center}
		$ \phi \coloncolonequals p~|~\top~|~\phi_1 \wedge \phi_2~|~\phi_1 \vee
	\phi_2~|~\Box \phi $
	\end{center}
	definierten Menge modaler Formeln, dann $S$ Erfülltheit von $\phi$ bewahrt,
	d.h. aus $xSy$ und $\kripke{M}_1, x \vDash \phi$ folgt $\kripke{M}_2, y
	\vDash \phi$.
}







\newcommand{\ex}{\exists R.}
\newcommand{\and}{\sqcap}

\aufgabe{2}{Subsumption in $\el$}
\angabe{%
	Wenden Sie den Subsumptionsalgorithmus für $\el$ an, um zu entscheiden, ob
	folgende Subsumptionen gültig sind:
\begin{equation}
	A \and B \and \ex (B \and C) \and \ex ( B \and C
	\and \ex D)
	\sqsubseteq
	A \and \ex (B \and \ex D) \and \ex ( C \and \ex D)
\end{equation}

\begin{equation}
	A \and \ex ( B \and \ex C) \and \ex (B \and \ex D)
	\sqsubseteq
	A \and \ex (B \and \ex (C \and D))
\end{equation}
}

Da jede Kante im Beschreibungsbaum die Relation $R$ darstellt, wird sie der
Einfachkeit halber weggelassen. \\
Um zu zeigen, ob eine Subsumption gültig ist, reicht es zu zeigen, dass es
eine Simulation von der rechten auf die linke Seite des Subsumptionszeichens
gibt.

\begin{center}
\begin{minipage}{0.4\textwidth}
\begin{tikzpicture}
	[edge from parent/.style={->,draw}]
	\node{A,B}
		child{node{B,C}}
		child{node{B,C}
			child {node{D}}
		};
\end{tikzpicture}
\end{minipage}
\begin{minipage}{0.4\textwidth}
\begin{tikzpicture}
	[edge from parent/.style={->,draw}]
	\node{A}
		child{node{B}
			child{node{D}}}
		child{node{C}
			child{node{D}}
		};
\end{tikzpicture}
\end{minipage}
\end{center}

An der Wurzel des rechten Beschreibungsbaums ist nur $A$ erfüllt. Links ist
zusätzlich noch $B$ erfüllt. Der erste Schritt ist also schon getan. Als
nächstes kann man links jeden Pfad der rechten Seite abdecken, indem man links
stets den rechten Zweig wählt. Nach dem ersten Schritt werden im linken
Beschreibungsbaum sowohl $B$ als auch $C$ erfüllt. Damit ist es egal in welche
Richtung man im rechten Baum geht, da hier nur eines der beiden Labels erfüllt
sein muss. Und auch der letzte Schritt zu $D$ ist mit dem rechten Zweig im
linken Baum zu simulieren. Damit existiert eine Simulation von rechts nach links
und die Subsumption ist gültig.

\begin{center}
\begin{minipage}{0.4\textwidth}
\begin{tikzpicture}
	[edge from parent/.style={->,draw}]
	\node{A}
		child{node{B}
			child{node{C}}}
		child{node{B}
			child{node{D}}
		};
\end{tikzpicture}
\end{minipage}
\begin{minipage}{0.4\textwidth}
\begin{tikzpicture}
	[edge from parent/.style={->,draw}]
	\node{A}
		child{node{B}
			child{node{C,D}}
		};
\end{tikzpicture}
\end{minipage}
\end{center}

In beiden Bäumen lässt sich der erste Schritt machen. Allerdings ist der zweite
Schritt im linken Baum unter dem Gesichtspunkt einer Simulation nicht möglich,
da entweder nur $C$ oder $D$ erfüllt werden. Der rechte Beschreibungsbaum
fordert jedoch, dass $C$ und $D$ erfüllt sein müssen. Damit ist diese
Subsumption ungültig.
