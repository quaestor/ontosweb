% vim: set filetype=tex:
\aufgabe{1}{Modellierung terminologischen Wissens}

\newcommand{\defin}[2]{#1 $\sqsubseteq$ #2}
\newcommand{\ando}{$\sqcap$\hspace{1pt}}
\newcommand{\kripke}[1]{\ensuremath{\mathcal{#1}}}
\newcommand{\relation}[1]{\ensuremath{\mathcal{#1}}}

Atomare Konzepte:
\begin{itemize}
	\item Album
	\item Song
	\item Genere, und diverse Subgenres
	\item Single
	\item Compilation
\end{itemize}

SuccessfulSong $\sqsubseteq$ Song 	$\sqcap$ $\exists$\hspace{1pt}appearsOn.Album
									$\sqcap$ $\exists$\hspace{1pt}appearsOn.Single
									$\sqcap$
									$\exists$\hspace{1pt}appearsOn.Compilation\\

SuccessfulArtist $\sqsubseteq$ Artist $\sqcap$
$\exists$\hspace{1pt}hasWritten.successfulSong\\

Artist $\sqsubseteq$ $\exists$\hspace{1pt}hasWritten.Song\\

Was nicht geht:
\begin{itemize}
	\item Beschreiben, dass ein Artist Songs in versch. Genres geschrieben hat.
	\item Zählen (z. B. eine Definition eines erfolgreichen Artists durch die
		Anzahl der Veröffentlichungen)
	\item \defin{BestOf}{Compilation \ando
		$\forall$contains.$\forall$writtenBy.\{\textit{sameauthor}\}}
	\item Selbstreferenzen
\end{itemize}

\aufgabe{2}{Verwendung von Werkzeugen}

\aufgabe{3}{Schließen in Modallogik}
Zz: Wenn der Label oberhalb des Regelstrichs gültig ist, so ist auch der Label
unterhalb des Regelstrichs gültig.



Es gilt: $\forall x \in M.\kripke{M},x\vDash \neg \phi_1,\dots,\neg \phi_n,\phi_0$
Dies heißt, dass entweder \[\kripke{M},x \vDash \phi_0\] oder \[\kripke{M},x
\nvDash \phi_0 \wedge \exists 0 < i \leq n.\kripke{M},x_j \nvDash \phi_i\] gelten muss.

Wir wählen nun ein beliebiges, aber festes, $x$ und betrachten dessen
Nachfolger:
\begin{enumerate}
	\item Für $x$ existiert kein Nachfolger, dann gilt trivialerweise für $x$
		$\Box \phi_0$.
	\item $x$ besitzt $k$ Nachfolger ($x_1, \dots, x_k$)
		\begin{enumerate}
			\item Wenn für alle Nachfolger $x_j$ ($j \in \{1, \dots, k\}$)
				\kripke{M}, $x_j \vDash \phi_0$
				Dann gilt:
				\[\kripke{M}, x \vDash \Box \phi_0\]
			\item Wenn für mindestens einen Nachfolger $x_j$ ($j \in \{1,
				\dots,k\}$)\kripke{M},$x_j \nvDash \phi_0$ gilt $\exists 0 < i \leq n.\kripke{M},x_j \nvDash
				\phi_i$, da für $x_j$ ebenfalls der Label oberhalb des
				Regelstriches gültig ist. Damit existiert ein Nachfolger von
				$x$, der nicht $\phi_i$ für ein $i$ erfüllt. Somit gilt $\neg
				\Box \phi_i$.
		\end{enumerate}
\end{enumerate}
Da $x$ beliebig ist die Behauptung korrekt.

\aufgabe{4}{(Keine) Bisimilarität}
\begin{center}
\begin{minipage}{0.4\textwidth}
\begin{tikzpicture}[->,>=stealth',shorten >=5pt, shorten <=5pt,auto,node distance=1.5cm,
  main node/.style={circle,inner sep=2pt,minimum size=1pt,fill,=black,draw,font=\sffamily\Large\bfseries,label=east:P},
  not main node/.style={circle,inner sep=2pt,minimum size=1pt,fill=black,draw,font=\sffamily\Large\bfseries,label=east:¬P}]

  \node[main node,label=west:$x_1$] (1) {};
  \node[main node,label=west:$x_2$] (2) [below of=1] {};
  \node[main node,label=west:$x_{31}$] (3) [below left of=2] {};
  \node[not main node,label=west:$x_{32}$] (4) [below right of=2] {};

  \path[every node/.style={font=\sffamily\small}]
    (1) edge node [right] {} (2)

	(2) edge node [right] {} (3)
        edge node [right] {} (4);
\end{tikzpicture}
\end{minipage}
\begin{minipage}{0.4\textwidth}
\begin{tikzpicture}[->,>=stealth',shorten >=5pt, shorten <=5pt,auto,node distance=1.5cm,
  main node/.style={circle,inner sep=2pt,minimum size=1pt,fill,=black,draw,font=\sffamily\Large\bfseries,label=0:P},
  not main node/.style={circle,inner sep=2pt,minimum size=1pt,fill=black,draw,font=\sffamily\Large\bfseries,label=0:¬P}]

  \node[main node,label=west:$y_1$] (1) {};
  \node[main node,label=west:$y_{21}$] (2) [below left of=1] {};
  \node[main node,label=west:$y_{31}$] (3) [below of=2] {};
  \node[main node,label=west:$y_{22}$] (4) [below right of=1] {};
  \node[not main node,label=west:$y_{32}$] (5) [below of=4] {};

  \path[every node/.style={font=\sffamily\small}]
    (1) edge node [right] {} (2)
		edge node [left]  {} (4)
	(2) edge node [right] {} (3)
	(4) edge node [right] {} (5);
\end{tikzpicture}
\end{minipage}
\end{center}

Gegeben sind die beiden baumförmigen Kripkemodelle \kripke{M} (links) und
\kripke{N} (rechts).
Diese sind bisimilar, wenn eine Relation $S \subseteq M \times N$ existiert und
diese sowie ihre Umkehrrelation $S^-$ Simulationen sind.

Zunächst zeigen wir, dass eine solche Relation $S$ nur existieren kann, wenn
$x_1 S y_1$ gilt. Gilt nämlich oBdA. $x_1 S y_i, i \neq 1$ dann folgt durch
Betrachtung der Nachfolger, wie nach Definitionsteil (ii), dass auf der rechten
Seite nicht ausrechend Nachfolger vorhanden sind.

Beweisführung durch Überprüfung der Definition von Simulation:

Nicht Modal:

Es gilt: $\kripke{M}, x_1 \vDash P$ und $\kripke{N},y_1 \vDash P$ (i). Für alle
Nachfolger von $x_1$, $x_2$, muss nun gelten: $\exists w.y_1 R w \wedge x_2 S w$
(ii).
\begin{enumerate}
	\item $w = y_{21}$ erfüllt diese Bedingung, da in beiden Punkten $P$ gilt.
		Überprüfe nun erneut (ii) bei dem Nachfolger $x_{32}$ von $x_2$. Nach
		Definition muss wiederum gelten: $\exists w'.y_{21} R w' \wedge x_{32} S
		w'$. Im Modell \kripke{N} kann nur $w' = y_{31}$ gelten. Damit ist $S$
		keine Simulation, da $\kripke{M}, x_{32} \nvDash P$ und $\kripke{N},
		y_{32} \vDash P$
	\item Für $w = y_{22}$ ist der Beweis analog bei Wahl Nachfolger $x_{31}$
		von $x_2$.
\end{enumerate}

Im modalen Fall erfolgt der Beweis analog mithilfe des
Bisimulationsinvarianzsatzes, wenn $P$ eine modale Formel ist.

